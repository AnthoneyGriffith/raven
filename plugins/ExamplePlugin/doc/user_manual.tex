%
\documentclass[pdf,12pt]{article}

%\usepackage{times}
%\usepackage[FIGBOTCAP,normal,bf,tight]{subfigure}
\usepackage{amsmath}
\usepackage{amssymb}
%\usepackage{pifont}
\usepackage{enumerate}
\usepackage{listings}
\usepackage{fullpage}
\usepackage{xcolor}          % Using xcolor for more robust color specification
%\usepackage{ifthen}          % For simple checking in newcommand blocks
%\usepackage{textcomp}
%\usepackage{authblk}         % For making the author list look prettier
%\renewcommand\Authsep{,~\,}

% Custom colors
\definecolor{deepblue}{rgb}{0,0,0.5}
\definecolor{deepred}{rgb}{0.6,0,0}
\definecolor{deepgreen}{rgb}{0,0.5,0}
\definecolor{forestgreen}{RGB}{34,139,34}
\definecolor{orangered}{RGB}{239,134,64}
\definecolor{darkblue}{rgb}{0.0,0.0,0.6}
\definecolor{gray}{rgb}{0.4,0.4,0.4}

\lstset {
  basicstyle=\ttfamily,
  frame=single
}

\lstdefinestyle{XML} {
    language=XML,
    extendedchars=true,
    breaklines=true,
    breakatwhitespace=true,
%    emph={name,dim,interactive,overwrite},
    emphstyle=\color{red},
    basicstyle=\ttfamily,
%    columns=fullflexible,
    commentstyle=\color{gray}\upshape,
    morestring=[b]",
    morecomment=[s]{<?}{?>},
    morecomment=[s][\color{forestgreen}]{<!--}{-->},
    keywordstyle=\color{cyan},
    stringstyle=\ttfamily\color{black},
    tagstyle=\color{darkblue}\bf\ttfamily,
    morekeywords={name,type},
%    morekeywords={name,attribute,source,variables,version,type,release,x,z,y,xlabel,ylabel,how,text,param1,param2,color,label},
}
\lstset{language=xml}

\usepackage{titlesec}
\newcommand{\sectionbreak}{\clearpage}
\setcounter{secnumdepth}{4}


%%%%%%%% Begin comands definition to input python code into document
\usepackage[utf8]{inputenc}

% Default fixed font does not support bold face
\DeclareFixedFont{\ttb}{T1}{txtt}{bx}{n}{9} % for bold
\DeclareFixedFont{\ttm}{T1}{txtt}{m}{n}{9}  % for normal

\usepackage{listings}

% Python style for highlighting
%\newcommand\pythonstyle{\lstset{
%language=Python,
%basicstyle=\ttm,
%otherkeywords={self, none, return},             % Add keywords here
%keywordstyle=\ttb\color{deepblue},
%emph={MyClass,__init__},          % Custom highlighting
%emphstyle=\ttb\color{deepred},    % Custom highlighting style
%stringstyle=\color{deepgreen},
%frame=tb,                         % Any extra options here
%showstringspaces=false            %
%}}


% Python environment
%\lstnewenvironment{python}[1][]
%{
%$\pythonstyle
%\lstset{#1}
%}
%{}

% Python for external files
%\newcommand\pythonexternal[2][]{{
%\pythonstyle
%\lstinputlisting[#1]{#2}}}
%
%\lstnewenvironment{xml}
%{}
%{}

% Python for inline
%\newcommand\pythoninline[1]{{\pythonstyle\lstinline!#1!}}

%\def\DRAFT{} % Uncomment this if you want to see the notes people have been adding
% Comment command for developers (Should only be used under active development)
%\ifdefined\DRAFT
%  \newcommand{\nameLabeler}[3]{\textcolor{#2}{[[#1: #3]]}}
%\else
%  \newcommand{\nameLabeler}[3]{}
%\fi
% Commands for making the LaTeX a bit more uniform and cleaner
%\newcommand{\TODO}[1]    {\textcolor{red}{\textit{(#1)}}}
\newcommand{\xmlAttrRequired}[1] {\textcolor{red}{\textbf{\texttt{#1}}}}
\newcommand{\xmlAttr}[1] {\textcolor{cyan}{\textbf{\texttt{#1}}}}
\newcommand{\xmlNodeRequired}[1] {\textcolor{deepblue}{\textbf{\texttt{<#1>}}}}
\newcommand{\xmlNode}[1] {\textcolor{darkblue}{\textbf{\texttt{<#1>}}}}
\newcommand{\xmlString}[1] {\textcolor{black}{\textbf{\texttt{'#1'}}}}
\newcommand{\xmlDesc}[1] {\textbf{\textit{#1}}} % Maybe a misnomer, but I am
                                                % using this to detail the data
                                                % type and necessity of an XML
                                                % node or attribute,
                                                % xmlDesc = XML description
\newcommand{\default}[1]{~\\*\textit{Default: #1}}
\newcommand{\nb} {\textcolor{deepgreen}{\textbf{~Note:}}~}

% The bm package provides \bm for bold math fonts.  Apparently
% \boldsymbol, which I used to always use, is now considered
% obsolete.  Also, \boldsymbol doesn't even seem to work with
% the fonts used in this particular document...
\usepackage{bm}

% Define tensors to be in bold math font.
\newcommand{\tensor}[1]{{\bm{#1}}}

% Override the formatting used by \vec.  Instead of a little arrow
% over the letter, this creates a bold character.
\renewcommand{\vec}{\bm}

% Define unit vector notation.  If you don't override the
% behavior of \vec, you probably want to use the second one.
\newcommand{\unit}[1]{\hat{\bm{#1}}}

% Use this to refer to a single component of a unit vector.
\newcommand{\scalarunit}[1]{\hat{#1}}

% \toprule, \midrule, \bottomrule for tables
\usepackage{booktabs}

% \llbracket, \rrbracket
\usepackage{stmaryrd}

\usepackage{hyperref}
\hypersetup{
    colorlinks,
    citecolor=black,
    filecolor=black,
    linkcolor=black,
    urlcolor=black
}

% Compress lists of citations like [33,34,35,36,37] to [33-37]
\usepackage{cite}

% If you want to relax some of the SAND98-0730 requirements, use the "relax"
% option. It adds spaces and boldface in the table of contents, and does not
% force the page layout sizes.
% e.g. \documentclass[relax,12pt]{SANDreport}
%
% You can also use the "strict" option, which applies even more of the
% SAND98-0730 guidelines. It gets rid of section numbers which are often
% useful; e.g. \documentclass[strict]{SANDreport}

% The INLreport class uses \flushbottom formatting by default (since
% it's intended to be two-sided document).  \flushbottom causes
% additional space to be inserted both before and after paragraphs so
% that no matter how much text is actually available, it fills up the
% page from top to bottom.  My feeling is that \raggedbottom looks much
% better, primarily because most people will view the report
% electronically and not in a two-sided printed format where some argue
% \raggedbottom looks worse.  If we really want to have the original
% behavior, we can comment out this line...
\raggedbottom
\setcounter{secnumdepth}{5} % show 5 levels of subsection
\setcounter{tocdepth}{5} % include 5 levels of subsection in table of contents

% ---------------------------------------------------------------------------- %
%
% Set the title, author, and date
%
\title{'CashFlow' User Manual \\
        \large Economics plugin for RAVEN \\
}
%\author{%
%\begin{tabular}{c} Author 1 \\ University1 \\ Mail1 \\ \\
%Author 3 \\ University3 \\ Mail3 \end{tabular} \and
%\begin{tabular}{c} Author 2 \\ University2 \\ Mail2 \\ \\
%Author 4 \\ University4 \\ Mail4\\
%\end{tabular} }


\author{
 \\Andrea Alfonsi\\
}

% There is a "Printed" date on the title page of a SAND report, so
% the generic \date should [WorkingDir:]generally be empty.
\date{\today}

%\def\component#1{\texttt{#1}}

% ---------------------------------------------------------------------------- %
%\newcommand{\systemtau}{\tensor{\tau}_{\!\text{SUPG}}}

% Added by Sonat
%\usepackage{placeins}
%\usepackage{array}

%\newcolumntype{L}[1]{>{\raggedright\let\newline\\\arraybackslash\hspace{0pt}}m{#1}}
%\newcolumntype{C}[1]{>{\centering\let\newline\\\arraybackslash\hspace{0pt}}m{#1}}
%\newcolumntype{R}[1]{>{\raggedleft\let\newline\\\arraybackslash\hspace{0pt}}m{#1}}

% end added by Sonat
% ---------------------------------------------------------------------------- %
%
% Start the document
%

\begin{document}
    \maketitle

    % ------------------------------------------------------------------------ %
    % The table of contents and list of figures and tables
    % Comment out \listoffigures and \listoftables if there are no
    % figures or tables. Make sure this starts on an odd numbered page
    %
    \cleardoublepage		% TOC needs to start on an odd page
    \tableofcontents
    %\listoffigures
    %\listoftables
    % ---------------------------------------------------------------------- %

    % ---------------------------------------------------------------------- %
    % This is where the body of the report begins; usually with an Introduction
    %
    \section{Introduction}
\label{sec:Introduction}
In this section, the developer of an ExternalModel plugin in RAVEN ( \cite{RAVEN} and \cite{RAVENtheoryMan}) should
report a brief description of what the plugin is intended to do.

\subsection{Example Plugin: SumOfExponential}
 In order to show how to develop a plugin, a simple Model that performs a summation
 of exponential (over time (or any monotonic variable)) has been developed:
 \newline
 \begin{math}
        Xi(t)=\sum_{i=1}^{n} coef_i*e^{var_i*t}
  \end{math}
  \newline
 where:
 \begin{itemize}
    \item $var_i$ is the $i-th$ variable (sampled by RAVEN). It is used as exponent of the 
            $exp$ function
   \item $coef_i$ is the $i-th$  coefficient for the exponential having as exponent the  
            variable $var_i$
   \item $t$ is the independent variable (monotonic variable).
 \end{itemize}



    \section{Input of Example Plugin for RAVEN}
In this section the developer of the ExternalModel plugin should report information
on how to use it in a RAVEN calculation. As mentioned in the previous section a simple 
Model that performs a summation of exponential (over time (or any monotonic variable) has been developed:
\newline
 \begin{math}
        Xi(t)=\sum_{i=1}^{n} coef_i*e^{var_i*t}
  \end{math}
  \newline
In this section, the input of the plugins must be reported. As an example, we report the
input of the Example Plugin ``SumOfExponential'' that we use as template.

\subsection{Input of SumOfExponential ExternalModel plugin}

The input of SumOfExponential is an XML file. An example of the input structure is given in Listing \ref{lst:InputExample}. The following section will discuss the
 different keywords in the input and describe how they are used in the SumOfExponential plugin.

\begin{lstlisting}[style=XML,morekeywords={anAttribute},caption=SumOfExponential 
  input example., label=lst:InputExample]
  <ExternalModel name="a_name" subType="SumOfExponential">
      <variables> Xi, monotonicVariable, var1, var2, var3</variables>
      <!-- xml portion for this plugin only -->
      <outputVariable>
        Xi
      </outputVariable>
      <monotonicVariable>
        time
      </monotonicVariable>
      <startMonotonicVariableValue>
        0.0
      </startMonotonicVariableValue>
      <endMonotonicVariableValue>
        1e6
      </endMonotonicVariableValue>
      <numberCalculationPoints>
        1000000
      </numberCalculationPoints>
      <coefficient varName="var1">1.1</coefficient>
      <coefficient varName="var2">-1.1</coefficient>
      <coefficient varName="var3">-1.1</coefficient>
 </ExternalModel>
\end{lstlisting}

As one can see, all the specifications of the SumOfExponential plugin are given in the 
\xmlNode{ExternalModel} block. Inside the \xmlNode{ExternalModel} block,  the XML
nodes that belongs to this plugin only (and not to the ExternalModel) are:
\begin{itemize}
  \item  \xmlNode{outputVariable}, \xmlDesc{string,
  required parameter}, the name of the output variable (e.g. $Xi$)
  \item  \xmlNode{monotonicVariable}, \xmlDesc{string,
  required parameter},  the name of the monotonic variable (e.g. $time$)
  \item  \xmlNode{startMonotonicVariableValue}, \xmlDesc{float,
  required parameter}, the starting value of the monotonic variable (e.g. time)
  \item  \xmlNode{endMonotonicVariableValue}, \xmlDesc{float,
  required parameter}, the ending value of the monotonic variable (e.g. time)
  \item  \xmlNode{numberCalculationPoints},\xmlDesc{int,
  required parameter}, the number of steps in the calculation (e.g. number of time 
  steps).
  \item  \xmlNode{coefficient}, \xmlDesc{float,
  optional parameter}, the $i-th$ coefficient for the exponential function ($coef_i$).
  Default value is $1.0$.The user can input a coefficient for each variable  of the model. 
  The mapping between the $coef_i$ and the associated variable is defined by the 
  attribute $varName$:
  \begin{itemize}
    \item \xmlAttr{varName}, \xmlDesc{required string attribute}, variable this coefficient
    is linked to.
  \end{itemize}
\end{itemize}



    \section*{Document Version Information}
    This document has been compiled using the following version of the plug-in git repository:
    \newline
    \input{version.tex}

    % ---------------------------------------------------------------------- %
    % References
    %
    \clearpage
    % If hyperref is included, then \phantomsection is already defined.
    % If not, we need to define it.
    \providecommand*{\phantomsection}{}
    \phantomsection
    \addcontentsline{toc}{section}{References}
    \bibliographystyle{ieeetr}
    \bibliography{user_manual}


    % ---------------------------------------------------------------------- %

\end{document}
