\subsection{Dependencies and Limitations}
The software should be designed with the fewest possible constraints. 
Ideally the software should run on a wide variety of evolving hardware, 
so it should follow well-adopted standards and guidelines. The software
 should run on any POSIX compliant system (including Windows POSIX 
 emulators such as MinGW). The software will also make use of artificial 
 intelligence and numerical libraries that run on POSIX systems as well. 
 The main interface for the software will be command line based with no 
 assumptions requiring advanced terminal capabilities such as coloring and line control. 
 \\In order to be functional, RAVEN depends on the following software/libraries.
\input{dependencies.tex}

\section{References}

\begin{itemize}

  \item ASME NQA 1 2008 with the NQA-1a-2009 addenda, ``Quality Assurance Requirements for Nuclear Facility Applications,'' First Edition, August 31, 2009.
  \item ISO/IEC/IEEE 24765:2010(E), ``Systems and software engineering Vocabulary,'' First Edition, December 15, 2010.
  \item LWP 13620, ``Managing Information Technology Assets''
\end{itemize}


\section{Definitions and Acronyms}

\subsection{Definitions}
\begin{itemize}
  \item \textbf{Baseline.} A specification or product (e.g., project plan, maintenance and operations [M\&O] plan, requirements, or 
design) that has been formally reviewed and agreed upon, that thereafter serves as the basis for use and further 
development, and that can be changed only by using an approved change control process. [ASME NQA-1-2008 with the 
NQA-1a-2009 addenda edited]
  \item \textbf{Validation.} Confirmation, through the provision of objective evidence (e.g., acceptance test), that the requirements 
for a specific intended use or application have been fulfilled. [ISO/IEC/IEEE 24765:2010(E) edited]
  \item \textbf{Verification.}
  \begin{itemize}
     \item The process of evaluating a system or component to determine whether the products of a given development 
     phase satisfy the conditions imposed at the start of that phase.
     \item  Formal proof of program correctness (e.g., requirements, design, implementation reviews, system tests). 
     [ISO/IEC/IEEE 24765:2010(E) edited]
  \end{itemize}
\end{itemize}

\subsection{Acronyms}
\begin{description}
\item[API] Application Programming Interfaces
\item[ASME] American Society of Mechanical Engineers
\item[CDF]  Comulative Distribution Functions
\item[DET] Dynamic Event Tree
\item[DOE] Department of Energy
\item[HDF5] Hierarchical Data Format (5)
\item[LWRS] Light Water Reactor Sustainability
\item[NEAMS] Nuclear Energy Advanced Modeling and Simulation
\item[NHES] Nuclear-Renewable Hybrid Energy Systems 
\item[INL] Idaho National Laboratory
\item[IT] Information Technology
\item[M\&O] Maintenance and Operations
\item[MC] Monte Carlo
\item[MOOSE] Multiphysics Object Oriented Simulation Environment
\item[NQA] Nuclear Quality Assurance
\item[POSIX]  Portable Operating System Interface
\item[PDF]  Probability Distribution (Density) Functions
\item[PP]  Post-Processor
\item[PRA]  Probabilistic Risk Assessment
\item[QA] Quality Assurance
\item[RAVEN] Risk Analysis and Virtual ENviroment
\item[ROM] Reduced Order Model
\item[SDD] System Design Description
\item[XML] eXtensible Markup Language 
\end{description}