%
% This is an example LaTeX file which uses the SANDreport class file.
% It shows how a SAND report should be formatted, what sections and
% elements it should contain, and how to use the SANDreport class.
% It uses the LaTeX article class, but not the strict option.
% ItINLreport uses .eps logos and files to show how pdflatex can be used
%
% Get the latest version of the class file and more at
%    http://www.cs.sandia.gov/~rolf/SANDreport
%
% This file and the SANDreport.cls file are based on information
% contained in "Guide to Preparing {SAND} Reports", Sand98-0730, edited
% by Tamara K. Locke, and the newer "Guide to Preparing SAND Reports and
% Other Communication Products", SAND2002-2068P.
% Please send corrections and suggestions for improvements to
% Rolf Riesen, Org. 9223, MS 1110, rolf@cs.sandia.gov
%
\documentclass[pdf,12pt]{INLreport}
% pslatex is really old (1994).  It attempts to merge the times and mathptm packages.
% My opinion is that it produces a really bad looking math font.  So why are we using it?
% If you just want to change the text font, you should just \usepackage{times}.
% \usepackage{pslatex}
\usepackage{times}
\usepackage[FIGBOTCAP,normal,bf,tight]{subfigure}
\usepackage{amsmath}
\usepackage{amssymb}
\usepackage{pifont}
\usepackage{enumerate}
\usepackage{listings}
\usepackage{fullpage}
\usepackage{xcolor}          % Using xcolor for more robust color specification
\usepackage{ifthen}          % For simple checking in newcommand blocks
\usepackage{textcomp}
\usepackage{graphicx}
\usepackage{float}
\usepackage[toc,page]{appendix}
%\usepackage{authblk}         % For making the author list look prettier
%\renewcommand\Authsep{,~\,}

% Custom colors
\definecolor{deepblue}{rgb}{0,0,0.5}
\definecolor{deepred}{rgb}{0.6,0,0}
\definecolor{deepgreen}{rgb}{0,0.5,0}
\definecolor{forestgreen}{RGB}{34,139,34}
\definecolor{orangered}{RGB}{239,134,64}
\definecolor{darkblue}{rgb}{0.0,0.0,0.6}
\definecolor{gray}{rgb}{0.4,0.4,0.4}

\lstset {
  basicstyle=\ttfamily,
  frame=single
}

\setcounter{secnumdepth}{5}
\lstdefinestyle{XML} {
    language=XML,
    extendedchars=true,
    breaklines=true,
    breakatwhitespace=true,
%    emph={name,dim,interactive,overwrite},
    emphstyle=\color{red},
    basicstyle=\ttfamily,
%    columns=fullflexible,
    commentstyle=\color{gray}\upshape,
    morestring=[b]",
    morecomment=[s]{<?}{?>},
    morecomment=[s][\color{forestgreen}]{<!--}{-->},
    keywordstyle=\color{cyan},
    stringstyle=\ttfamily\color{black},
    tagstyle=\color{darkblue}\bf\ttfamily,
    morekeywords={name,type},
%    morekeywords={name,attribute,source,variables,version,type,release,x,z,y,xlabel,ylabel,how,text,param1,param2,color,label},
}
\lstset{language=python,upquote=true}

\usepackage{titlesec}
\newcommand{\sectionbreak}{\clearpage}
\setcounter{secnumdepth}{4}

%\titleformat{\paragraph}
%{\normalfont\normalsize\bfseries}{\theparagraph}{1em}{}
%\titlespacing*{\paragraph}
%{0pt}{3.25ex plus 1ex minus .2ex}{1.5ex plus .2ex}

%%%%%%%% Begin comands definition to input python code into document
\usepackage[utf8]{inputenc}

% Default fixed font does not support bold face
\DeclareFixedFont{\ttb}{T1}{txtt}{bx}{n}{9} % for bold
\DeclareFixedFont{\ttm}{T1}{txtt}{m}{n}{9}  % for normal

\usepackage{listings}

% Python style for highlighting
\newcommand\pythonstyle{\lstset{
language=Python,
basicstyle=\ttm,
otherkeywords={self, none, return},             % Add keywords here
keywordstyle=\ttb\color{deepblue},
emph={MyClass,__init__},          % Custom highlighting
emphstyle=\ttb\color{deepred},    % Custom highlighting style
stringstyle=\color{deepgreen},
frame=tb,                         % Any extra options here
showstringspaces=false            %
}}


% Python environment
\lstnewenvironment{python}[1][]
{
\pythonstyle
\lstset{#1}
}
{}

% Python for external files
\newcommand\pythonexternal[2][]{{
\pythonstyle
\lstinputlisting[#1]{#2}}}

\lstnewenvironment{xml}
{}
{}

% Python for inline
\newcommand\pythoninline[1]{{\pythonstyle\lstinline!#1!}}

% Named Colors for the comments below (Attempted to match git symbol colors)
\definecolor{RScolor}{HTML}{8EB361}  % Sonat (adjusted for clarity)
\definecolor{DPMcolor}{HTML}{E28B8D} % Dan
\definecolor{JCcolor}{HTML}{82A8D9}  % Josh (adjusted for clarity)
\definecolor{AAcolor}{HTML}{8D7F44}  % Andrea
\definecolor{CRcolor}{HTML}{AC39CE}  % Cristian
\definecolor{RKcolor}{HTML}{3ECC8D}  % Bob (adjusted for clarity)
\definecolor{DMcolor}{HTML}{276605}  % Diego (adjusted for clarity)
\definecolor{PTcolor}{HTML}{990000}  % Paul

\def\DRAFT{} % Uncomment this if you want to see the notes people have been adding
% Comment command for developers (Should only be used under active development)
\ifdefined\DRAFT
  \newcommand{\nameLabeler}[3]{\textcolor{#2}{[[#1: #3]]}}
\else
  \newcommand{\nameLabeler}[3]{}
\fi
\newcommand{\alfoa}[1] {\nameLabeler{Andrea}{AAcolor}{#1}}
\newcommand{\cristr}[1] {\nameLabeler{Cristian}{CRcolor}{#1}}
\newcommand{\mandd}[1] {\nameLabeler{Diego}{DMcolor}{#1}}
\newcommand{\maljdan}[1] {\nameLabeler{Dan}{DPMcolor}{#1}}
\newcommand{\cogljj}[1] {\nameLabeler{Josh}{JCcolor}{#1}}
\newcommand{\bobk}[1] {\nameLabeler{Bob}{RKcolor}{#1}}
\newcommand{\senrs}[1] {\nameLabeler{Sonat}{RScolor}{#1}}
\newcommand{\talbpaul}[1] {\nameLabeler{Paul}{PTcolor}{#1}}
% Commands for making the LaTeX a bit more uniform and cleaner
\newcommand{\TODO}[1]    {\textcolor{red}{\textit{(#1)}}}
\newcommand{\xmlAttrRequired}[1] {\textcolor{red}{\textbf{\texttt{#1}}}}
\newcommand{\xmlAttr}[1] {\textcolor{cyan}{\textbf{\texttt{#1}}}}
\newcommand{\xmlNodeRequired}[1] {\textcolor{deepblue}{\textbf{\texttt{<#1>}}}}
\newcommand{\xmlNode}[1] {\textcolor{darkblue}{\textbf{\texttt{<#1>}}}}
\newcommand{\xmlString}[1] {\textcolor{black}{\textbf{\texttt{'#1'}}}}
\newcommand{\xmlDesc}[1] {\textbf{\textit{#1}}} % Maybe a misnomer, but I am
                                                % using this to detail the data
                                                % type and necessity of an XML
                                                % node or attribute,
                                                % xmlDesc = XML description
\newcommand{\default}[1]{~\\*\textit{Default: #1}}
\newcommand{\nb} {\textcolor{deepgreen}{\textbf{~Note:}}~}

%

%%%%%%%% End comands definition to input python code into document

%\usepackage[dvips,light,first,bottomafter]{draftcopy}
%\draftcopyName{Sample, contains no OUO}{70}
%\draftcopyName{Draft}{300}

% The bm package provides \bm for bold math fonts.  Apparently
% \boldsymbol, which I used to always use, is now considered
% obsolete.  Also, \boldsymbol doesn't even seem to work with
% the fonts used in this particular document...
\usepackage{bm}

% Define tensors to be in bold math font.
\newcommand{\tensor}[1]{{\bm{#1}}}

% Override the formatting used by \vec.  Instead of a little arrow
% over the letter, this creates a bold character.
\renewcommand{\vec}{\bm}

% Define unit vector notation.  If you don't override the
% behavior of \vec, you probably want to use the second one.
\newcommand{\unit}[1]{\hat{\bm{#1}}}
% \newcommand{\unit}[1]{\hat{#1}}

% Use this to refer to a single component of a unit vector.
\newcommand{\scalarunit}[1]{\hat{#1}}

% set method for expressing expected value as E[f]
\newcommand{\expv}[1]{\ensuremath{\mathbb{E}[ #1]}}

% \toprule, \midrule, \bottomrule for tables
\usepackage{booktabs}

% \llbracket, \rrbracket
\usepackage{stmaryrd}

\usepackage{hyperref}
\hypersetup{
    colorlinks,
    citecolor=black,
    filecolor=black,
    linkcolor=black,
    urlcolor=black
}

% Compress lists of citations like [33,34,35,36,37] to [33-37]
\usepackage{cite}

% If you want to relax some of the SAND98-0730 requirements, use the "relax"
% option. It adds spaces and boldface in the table of contents, and does not
% force the page layout sizes.
% e.g. \documentclass[relax,12pt]{SANDreport}
%
% You can also use the "strict" option, which applies even more of the
% SAND98-0730 guidelines. It gets rid of section numbers which are often
% useful; e.g. \documentclass[strict]{SANDreport}

% The INLreport class uses \flushbottom formatting by default (since
% it's intended to be two-sided document).  \flushbottom causes
% additional space to be inserted both before and after paragraphs so
% that no matter how much text is actually available, it fills up the
% page from top to bottom.  My feeling is that \raggedbottom looks much
% better, primarily because most people will view the report
% electronically and not in a two-sided printed format where some argue
% \raggedbottom looks worse.  If we really want to have the original
% behavior, we can comment out this line...
\raggedbottom
\setcounter{secnumdepth}{5} % show 5 levels of subsection
\setcounter{tocdepth}{5} % include 5 levels of subsection in table of contents

% ---------------------------------------------------------------------------- %
%
% Set the title, author, and date
%
\title{RAVEN Analytic Test Documentation}

\author{
\textbf{\textit{Principal Investigator (PI):}}
 \\Cristian Rabiti\\
\textbf{\textit{Main Developers:}}
\\Andrea Alfonsi
\\Joshua Cogliati
\\Diego Mandelli
\\Robert Kinoshita
\\Congjian Wang
\\Daniel P. Maljovec
\\Paul W. Talbot
}

% There is a "Printed" date on the title page of a SAND report, so
% the generic \date should [WorkingDir:]generally be empty.
\date{}


% ---------------------------------------------------------------------------- %
% Set some things we need for SAND reports. These are mandatory
%
%TODO someone help me know what goes here?  - Paul
\SANDnum{todo}
\SANDprintDate{todo}
\SANDauthor{todo}
\SANDreleaseType{todo}


% ---------------------------------------------------------------------------- %
% Include the markings required for your SAND report. The default is "Unlimited
% Release". You may have to edit the file included here, or create your own
% (see the examples provided).
%
% \include{MarkOUO} % Not needed for unlimted release reports

\def\component#1{\texttt{#1}}

% ---------------------------------------------------------------------------- %
\newcommand{\systemtau}{\tensor{\tau}_{\!\text{SUPG}}}

% ---------------------------------------------------------------------------- %
%
% Start the document
%

\begin{document}
    \maketitle

    \cleardoublepage		% TOC needs to start on an odd page
    \tableofcontents

    % ---------------------------------------------------------------------- %
    % This is where the body of the report begins; usually with an Introduction
    %
    \SANDmain		% Start the main part of the report

\section{Introduction}
RAVEN is a flexible and multi-purpose uncertainty quantification (UQ), regression analysis, probabilistic risk assessment 
(PRA), data analysis and model optimization software.  
Its broad spectrum of application determined the need of an integrated design (see RAVEN SDD document for  details)
of the software aimed to integrate multiple requirements.
\\This document is aimed to report and explain the RAVEN software requirements.

\section{Projectile (vacuum, gravity)}
Associated external model: \texttt{projectile.py}

Solves the projectile motion equations
\begin{equation}
  x = x_0 + v_{x,0} t,
\end{equation}
\begin{equation}
  y = y_0 + v_{y,0} t + \frac{a}{2} t^2,
\end{equation}
with the following inputs
\begin{itemize}
  \item $x_0$, or \texttt{x0}, initial horizontal position,
  \item $y_0$, or \texttt{y0}, initial vertical position,
  \item $v_0$, or \texttt{v0}, initial speed (scalar),
  \item $\theta$, or \texttt{ang}, angle with respect to horizontal plane,
\end{itemize}
and following responses:
\begin{itemize}
  \item $r$, or \texttt{r}, the horizontal distance traveled before hitting $y=0$,
  \item $x$, or \texttt{x}, the time-dependent horizontal position,
  \item $y$, or \texttt{y}, the time-dependent vertical position,
  \item $t$, or \texttt{time}, the series of time steps taken.
\end{itemize}
The simulation takes 10 equally spaced time steps from 0 to 1 second, inclusive, and returns all four values
as vector quantities.

\subsection{Grid, $x_0,y_0$}
If a Grid sampling strategy is used and the following distributions are applied to $x_0$ and $y_0$, with three
samples equally spaced on the CDF between 0.01 and 0.99 for each input, the following are some of the samples obtained.
\begin{itemize}
  \item $x_0$ is distributed normally with mean 0 and standard deviation 1,
  \item $y_0$ is distributed normally with mean 1 and standard deviation 0.2.
\end{itemize}

\begin{table}[h!]
  \centering
  \begin{tabular}{c c|c|c c}
    $x_0$ & $y_0$ & $t$ & $x$ & $y$ \\ \hline
    -2.32634787404 & 0.53473045192 & 0 & -2.32634787404 & 0.534730425192 \\
                   &               & $1/3$ & -2.09064561365 & 0.225988241143 \\
                   &               & 1 & -1.61924109285 & -3.65816279362 \\ \hline
    1.0            & 0.0           & 0 & 1.0 & 0.0 \\
                   &               & $1/3$ & 0.235702260396 & 0.691257815951 \\
                   &               & 1 & 0.707106781187 & -3.19289321881
  \end{tabular}
\end{table}

\section{Attenuation}
Associated external model: \texttt{attenuate.py}

Attenuation evaluation for quantity of interest $u$ with input parameters $Y=[y_1,\ldots,y_N]$:
\begin{equation}
u(Y) = \prod_{n=1}^N e^{-y_n/N}.
\end{equation}
This is the solution to the exit strength of a monodirectional, single-energy beam of neutral particles
incident on a unit length material divided into $N$ sections with independently-varying absorption cross
sections.  This test is useful for its analytic statistical moments as well as difficulty to represent
exactly using polynomial representations.

\subsection{Uniform}
Let all $y_n$ be uniformly distributed between 0 and 1.  The first two statistical moments are:
\subsubsection{mean}
\begin{align}
\expv{u(Y)} &=\int_{0}^1 dY \rho(Y)u(Y),\notag \\
  &=\int_{0}^1 dy_1\cdots\int_{0}^1 dy_N \prod_{n=1}^N e^{-y_n/N},\notag \\
  &=\left[ \int_{0}^1 dy e^{-y/N}\right]^N,\\
  &=\left[\left(-Ne^{-y/N}\right)\bigg|_0^1\right]^N,\notag \\
  &=\left[N\left(1-e^{-1/N}\right)\right]^N. \notag
\end{align}
\subsubsection{variance}
\begin{align}
\expv{u(Y)^2} &= \int_{0}^1 dY \rho(Y)u(Y), \notag \\
  &=\int_{0}^1 dy_1\cdots\int_{0}^1 dy_N \frac{1}{1^N} \left(\prod_{n=1}^N e^{-y_n/N}\right)^2,\notag \\
  &=\left[\left(\int_{0}^1 dy\ e^{-2y/N} \right)\right]^N,\notag \\
  &=\left[\left(\frac{N}{2}e^{-2y/N} \right)\bigg|_{0}^1 \right]^N, \\
  &=\left[\frac{N}{2}\left(1-e^{-2/N}\right)\right]^N.\notag \\
\text{var}[u(Y)] &= \expv{u(Y)^2}-\expv{u(Y)}^2, \notag \\
  &= \left[\frac{N}{2}\left(1-e^{-2/N}\right)\right]^N - \left[N\left(1-e^{-1/N}\right)\right]^{2N}.
\end{align}
\subsubsection{numeric values}
Some numeric values for the mean and variance are listed below for several input cardinalities $N$.
\begin{table}[h!]
\centering
\begin{tabular}{c|c|c}
$N$ & mean & variance \\ \hline
2 & 0.61927248698470190 & 0.01607798775751018 \\
4 & 0.61287838657652779 & 0.00787849640356994 \\
6 & 0.61075635579491642 & 0.00520852933409887
\end{tabular}
\end{table}

\subsection{Multivariate Normal}
Let $Y$ be $N$-dimensional, and have a multivariate normal distribution:
\begin{equation}
Y \thicksim N(\mu,\Sigma)
\end{equation}
with $N$-dimensional mean vector $\mu=[\mu_{y_1},\mu_{y_2},\ldots,\mu_{y_N}]$, and $N X N$ covariance matrix:
\begin{equation}
\Sigma = [Cov[y_i,y_j]],i = 1,2,\ldots,N; j = 1,2,\ldots,N
\end{equation}

To be simplicity, we assume there are no correlations between the input parameters. Then, the covariance matrix can be written
as:
\begin{equation}
\Sigma =
\begin{pmatrix}
\sigma_{y_1}^2 & 0 &\ldots & 0 \\
0 & \sigma_{y_2}^2 &\ldots & 0 \\
\vdots &\vdots &\ddots & \vdots \\
0 & 0 & \ldots & \sigma_{y_N}^2\\
\end{pmatrix}
\end{equation}
where $\sigma_{y_i}^2 = Cov[y_i,y_i]$, for $i = 1,2,\ldots,N$. Based on this assumption, the first two statistical moments are:
\subsubsection{mean}
\begin{align}
\expv{u(Y)} &=\int_{-\infty}^\infty dY \rho(Y)u(Y) \notag \\
  &=\int_{-\infty}^\infty dy_1 (1/\sqrt{2 \pi \sigma_{y_1}}e^{-\frac{(y_1-\mu_{y_1})^2}{2\sigma_{y_1}^2}})\cdots\int_{-\infty}^\infty dy_N  (1/\sqrt{2 \pi \sigma_{y_N}}e^{-\frac{(y_N-\mu_{y_N})^2}{2\sigma_{y_N}^2}})\prod_{n=1}^N e^{-y_n/N} \\
  &=\prod_{n=1}^N e^{\frac{\sigma_{y_i}^2}{2n^2}-\frac{\mu_{y_i}}{n}}. \notag
\end{align}
\subsubsection{variance}
\begin{align}
\text{var}[u(Y)]=\expv{(u(Y)-\expv{u(Y)})^2} &= \int_{-\infty}^\infty dY \rho(Y)(u(Y)-\expv{u(Y)})^2 \notag \\
  &=\int_{-\infty}^\infty dy_1 (1/\sqrt{2 \pi \sigma_{y_1}}e^{-\frac{(y_1-\mu_{y_1})^2}{2\sigma_{y_1}^2}})\\
  &\cdots\int_{-\infty}^\infty dy_N  (1/\sqrt{2 \pi \sigma_{y_N}}e^{-\frac{(y_N-\mu_{y_N})^2}{2\sigma_{y_N}^2}})(\prod_{n=1}^N e^{-y_n/N}-\expv{u(Y)})^2 \notag\\
  &=\prod_{n=1}^N e^{\frac{2 \sigma_{y_i}^2}{n^2}-\frac{2\mu_{y_i}}{n}}. \notag
\end{align}
\subsubsection{numeric values}
For example, for given mean $\mu = [0.5, -0.4, 0.3, -0.2, 0.1]$, and covariance
\begin{equation}
\Sigma =
\begin{pmatrix}
0.64 & 0 & 0 & 0 & 0 \\
0 & 0.49 & 0 & 0 & 0 \\
0 & 0 & 0.09 & 0 & 0 \\
0 & 0 & 0 & 0.16 & 0 \\
0 & 0 & 0 & 0 & 0.25 \\
\end{pmatrix}
\end{equation}
The mean and variance can computed using previous equation, and the results are:
\begin{equation}
\expv{u(Y)} = 0.97297197488624509
\end{equation}
\begin{equation}
\text{var}{u(Y)} = 0.063779804051749989
\end{equation}

\subsection{Changing lower, upper bounds}
A parametric study can be made by changing the lower and upper bounds of the material opacities.

The objective is to determine the effects on the exit strength $u$ of a beam impinging on a
unit-length material subdivided into two materials with opacities $y_1, y_2$.  The range of values for
these opacities varies from lower bound $y_\ell$ to higher bound $y_h$, and the bounds are always
the same for both opacities.

We consider evaluating the lower and upper bounds
on a grid, and determine the expected values for the opacity means and exit strength.

The analytic values for the exit strength expected value depends on the lower and upper bound
as follows:
\begin{align}
  \bar u(y_1,y_2) &= \int_{y_\ell}^{y_h}\int_{y_\ell}^{y_h} \left(\frac{1}{y_h-y_\ell}\right)^2
    e^{-(y_1+y_2)/2} dy_1 dy_2, \\
    &= \frac{4e^{-y_h-y_\ell}\left(e^{y_h/2}-e^{y_\ell/2}\right)^2}{(y_h-y_\ell)^2}.
\end{align}

Numerically, the following grid points result in the following expected values:

\begin{table}[h!]
\centering
\begin{tabular}{c c|c|c}
$y_\ell$ & $y_h$ & $\bar y_1=\bar y_2$ & $\bar u$ \\ \hline
0.00 & 0.50 & 0.250 & 0.782865 \\
0.00 & 0.75 & 0.375 & 0.695381 \\
0.00 & 1.00 & 0.500 & 0.619272 \\
0.25 & 0.50 & 0.375 & 0.688185 \\
0.25 & 0.75 & 0.500 & 0.609696 \\
0.25 & 1.00 & 0.625 & 0.541564 \\
0.50 & 0.50 & 0.500 & 0.606531 \\
0.50 & 0.75 & 0.625 & 0.535959 \\
0.50 & 1.00 & 0.750 & 0.474832
\end{tabular}
\end{table}


\input{tensor_poly.tex}
\input{sobol_sens.tex}
\input{rims.tex}
\section{Parabolas}
Associated external model: \texttt{parabolas.py}

This model is a simple $N$-dimensional parabolic response $u(Y)$,
\begin{equation}
  u(Y) = \sum_{n=1}^N -y_n^2,
\end{equation}
where the uncertain inputs are $Y=(y_1,\cdots,y_N)$ and can be defined arbitrarily.  For optimization
searches, it is possible to obtain a maximum in the interior of the input by assuring the range of each
input variable include 0.  In this case, the maximum point will be ${0}^N$.

\begin{appendices}
  \section{Document Version Information}
  \input{../version.tex}
\end{appendices}


    % ---------------------------------------------------------------------- %
    % References
    %
    \clearpage
    % If hyperref is included, then \phantomsection is already defined.
    % If not, we need to define it.
    \providecommand*{\phantomsection}{}
    \phantomsection
    \addcontentsline{toc}{section}{References}
    \bibliographystyle{ieeetr}
    \bibliography{analytic_tests}

\end{document}
