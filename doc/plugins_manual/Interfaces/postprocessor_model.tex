\section{PostProcessor Plugins}
\label{sec:postprocessorPlugins}

Similar to other entities plugins, the PostProcessor plugin accepts RAVEN DataObject(s) as input(s),
perform additional operations on the data stored in the DataObject(s), and save the calculations
results into a new RAVEN DataObject. This procedure does not require modifying RAVEN itself, and the
developed plugin entity is going to be embedded in RAVEN at run-time.

At the initialization stage, RAVEN imports all the \texttt{PostProcessor} plugin entity that are contained in
\texttt{plugin/src} directory and performs some preliminary cross-checks.
\\It is important to notice that the name of class in the plugin entity module is the one
the user needs to specify when the new postprocessor plugin entity needs to be used.
For example, if the new plugin module contains the class ``NewPostProcessor'',
the \textit{subType} in the \xmlNode{PostProcessor} block will be \xmlAttr{PluginName.NewPostProcessor}:
\begin{lstlisting}[language=python, basicstyle=\scriptsize\ttfamily, breaklines=True, columns=fullflexible]
  from ravenframework.PluginBaseClasses.PostProcessorPluginBase import PostProcessorPluginBase
  class NewPostProcessor(PostProcessorPluginBase):
    ...
\end{lstlisting}
\begin{lstlisting}[style=XML,morekeywords={name,file}] %moreemph={name,file}]
  <Models>
    ...
    <PostProcessor name='whatever' subType='PluginName.NewPostProcessor'>
     ...
    </PostProcessor>
    ...
  </Models>
\end{lstlisting}

When loading an ``PostProcessor Plugin entities'', RAVEN expects to find, in the class
representing the plugin, the following required methods:
\begin{lstlisting}[language=python, basicstyle=\scriptsize\ttfamily, breaklines=True, columns=fullflexible]
  from ravenframework.PluginBaseClasses.CodePluginBase import CodePluginBase
  class NewCode(CodePluginBase):
    ...
    def run(self,inputIn):
      """
       This method to perform operations on the data from input DataObject(s)
       @ In, inputIn, dict, dictionary which contains the data inside the input DataObject
       @ Out, outputDict, dict, the output dictionary, passing through HistorySet info
      """
\end{lstlisting}

Where the input \texttt{inputIn} for \textbf{run} method has the following format:
\begin{lstlisting}[language=python, basicstyle=\scriptsize\ttfamily, breaklines=True, columns=fullflexible]
inputIn = {'Data':listData, 'Files':listOfFiles},
listData has the following format if 'xrDataset' is passed to self.setInputDataType('xrDataset')
(listOfInputVars, listOfOutVars, xr.Dataset)
Otherwise listData has the following format: (listOfInputVars, listOfOutVars, DataDict) with
DataDict is a dictionary that has the format:
    dataDict['dims']     = dict {varName:independentDimensions}
    dataDict['metadata'] = dict {metaVarName:metaVarValue}
    dataDict['type'] = str TypeOfDataObject
    dataDict['inpVars'] = list of input variables
    dataDict['outVars'] = list of output variables
    dataDict['numberRealization'] = int SizeOfDataObject
    dataDict['name'] = str DataObjectName
    dataDict['metaKeys'] = list of meta variables
    dataDict['data'] = dict {varName: varValue(1-D or 2-D numpy array)}
\end{lstlisting}

In addition, the following optional methods can be specified:
\begin{lstlisting}[language=python, basicstyle=\scriptsize\ttfamily, breaklines=True, columns=fullflexible]
  from ravenframework.PluginBaseClasses.CodePluginBase import CodePluginBase
  class NewCode(CodePluginBase):
    ...
    @classmethod
    def getInputSpecification(cls):
      """
        Method to get a reference to a class that specifies the input data for
        class cls.
        @ In, cls, the class for which we are retrieving the specification
        @ Out, inputSpecification, InputData.ParameterInput, class to use for
          specifying input of cls.
      """
    def __init__(self):
      """
        Constructor
        @ In, None
        @ Out, None
      """
    def initialize(self, runInfo, inputs, initDict=None):
      """
        Method to initialize the DataClassifier post-processor.
        @ In, runInfo, dict, dictionary of run info (e.g. working dir, etc)
        @ In, inputs, list, list of inputs
        @ In, initDict, dict, optional, dictionary with initialization options
        @ Out, None
      """
    def _handleInput(self, paramInput):
      """
        Function to handle the parameter input.
        @ In, paramInput, ParameterInput, the already parsed input.
        @ Out, None
      """
\end{lstlisting}

The explanation for all above optional methods can be found in section \ref{subsec:commonMethods}.
