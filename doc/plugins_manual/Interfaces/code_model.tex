\section{Code Plugins}
\label{sec:codePlugins}

The procedure of creating a new code/application plugin with RAVEN is a straightforward process.
The plugin is performed through a Python interface that interprets the information coming from RAVEN
and translates them into the input of the driven code plugin. This procedure does not require
modifying RAVEN itself. Instead, the developer creates a new Python Code Plugin entity that is
going to be embedded in RAVEN at run-time (no need to introduce  hard-coded coupling statements).

At the initialization stage, RAVEN imports all the code plugin entity that are contained in
\texttt{plugin/src} directory and performs some preliminary cross-checks.
\\It is important to notice that the name of class in the plugin entity module is the one
the user needs to specify when the new code plugin needs to be used.
For example, if the new plugin module contains the class ``NewCode'',
the \textit{subType} in the \xmlNode{Code} block will be \xmlAttr{PluginName.NewCode}:
\begin{lstlisting}[language=python, basicstyle=\scriptsize\ttfamily, breaklines=True, columns=fullflexible]
  from ravenframework.PluginBaseClasses.CodePluginBase import CodePluginBase
  class NewCode(CodePluginBase):
    ...
\end{lstlisting}
\begin{lstlisting}[style=XML,morekeywords={name,file}] %moreemph={name,file}]
  <Models>
    ...
    <Code name='whatever' subType='PluginName.NewCode'>
     ...
    </Code>
    ...
  </Models>
\end{lstlisting}

When loading an ``Code Plugin'', RAVEN expects to find, in the class
representing the plugin, the following required methods:
\begin{lstlisting}[language=python, basicstyle=\scriptsize\ttfamily, breaklines=True, columns=fullflexible]
  from ravenframework.PluginBaseClasses.CodePluginBase import CodePluginBase
  class NewCode(CodePluginBase):
    ...
    def generateCommand(self, inputFiles, executable, clargs=None, fargs=None, preExec=None):
      """
        See base class.  Collects all the clargs and the executable to produce the command-line call.
        Returns tuple of commands and base file name for run.
        Commands are a list of tuples, indicating parallel/serial and the execution command to use.
        @ In, inputFiles, list, List of input files (lenght of the list depends on the number of inputs have been added in the Step is running this code)
        @ In, executable, string, executable name with absolute path (e.g. /home/path_to_executable/code.exe)
        @ In, clargs, dict, optional, dictionary containing the command-line flags the user can specify in the input (e.g. under the node < Code >< clargstype =0 input0arg =0 i0extension =0 .inp0/ >< /Code >)
        @ In, fargs, dict, optional, a dictionary containing the axuiliary input file variables the user can specify in the input (e.g. under the node < Code >< clargstype =0 input0arg =0 aux0extension =0 .aux0/ >< /Code >)
        @ In, preExec, string, optional, a string the command that needs to be pre-executed before the actual command here defined
        @ Out, returnCommand, tuple, tuple containing the generated command. returnCommand[0] is the command to run the code (string), returnCommand[1] is the name of the output root
      """
    def createNewInput(self, currentInputFiles, oriInputFiles, samplerType, **Kwargs):
      """
        Generate a new Projectile input file (txt format) from the original, changing parameters
        as specified in Kwargs['SampledVars']. In addition, it creaes an additional input file including the vector data to be
        passed to Dymola.
        @ In, currentInputFiles, list,  list of current input files (input files from last this method call)
        @ In, oriInputFiles, list, list of the original input files
        @ In, samplerType, string, Sampler type (e.g. MonteCarlo, Adaptive, etc. see manual Samplers section)
        @ In, Kwargs, dictionary, kwarded dictionary of parameters. In this dictionary there is another dictionary called "SampledVars"
              where RAVEN stores the variables that got sampled (e.g. Kwargs['SampledVars'] => {'var1':10,'var2':40})
        @ Out, newInputFiles, list, list of newer input files, list of the new input files (modified and not)
      """

\end{lstlisting}
In addition, the following optional methods can be specified:
\begin{lstlisting}[language=python, basicstyle=\scriptsize\ttfamily, breaklines=True, columns=fullflexible]
  from ravenframework.PluginBaseClasses.CodePluginBase import CodePluginBase
  class NewCode(CodePluginBase):
    ...
    def __init__(self):
      """
        Constructor
        @ In, None
        @ Out, None
      """
    def initialize(self, runInfo, oriInputFiles):
      """
        Method to initialize the run of a new step
        @ In, runInfo, dict,  dictionary of the info in the <RunInfo> XML block
        @ In, oriInputFiles, list, list of the original input files
        @ Out, None
      """
    def _readMoreXML(self, xmlNode):
      """
        Function to read the portion of the xml input that belongs to this specialized class and
        initialize some members based on inputs. This can be overloaded in specialized code interface in order
        to read specific flags
        @ In, xmlNode, xml.etree.ElementTree.Element, Xml element node
        @ Out, None
      """
    def getInputExtension(self):
      """
        Return a tuple of possible file extensions for a simulation initialization file (e.g., input.i).
        @ In, None
        @ Out, validExtensions, tuple, tuple of valid extensions
      """
    def finalizeCodeOutput(self, command, output, workingDir):
      """
        Called by RAVEN to modify output files (if needed) so that they are in a proper form.
        In this case, the default .mat output needs to be converted to .csv output, which is the
        format that RAVEN can communicate with.
        @ In, command, string, the command used to run the just ended job
        @ In, output, string, the Output name root
        @ In, workingDir, string, current working dir
        @ Out, output, string, optional, present in case the root of the output file gets changed in this method.
      """
    def checkForOutputFailure(self, output, workingDir):
      """
        This method is called by RAVEN at the end of each run if the return code is == 0.
        This method needs to be implemented for the codes that, if the run fails, return a return code that is 0
        This can happen in those codes that record the failure of the job (e.g. not converged, etc.) as normal termination (returncode == 0)
        This method can be used, for example, to parse the output file looking for a special keyword that testifies that a particular job got failed
        (e.g. in RELAP5 would be the keyword "********")
        @ In, output, string, the Output name root
        @ In, workingDir, string, current working dir
        @ Out, failure, bool, True if the job is failed, False otherwise
      """
\end{lstlisting}

The explanation for all above required and optional methods except the \texttt{\_readMoreXML} can be found in
RAVEN user manual section \textbf{Advanced Users: How to couple a new code}. We recommand the plugin developers
take a look at the RAVEN user manual for the detailed information and learn how to use these methods. In the
following subsection, the \texttt{\_readMoreXML} method is explained.

\subsubsection{Method: \texttt{\_readMoreXML}}
\label{subsubsec:codePluginReadMoreXML}
The \textbf{\_readMoreXML} method can be implemented by the
Code plugin developer if the XML input that belongs to this Code
plugin needs to be extended to contain other information.
%
If this method is implemented in the \textbf{NewCodePlugin}, RAVEN is going to
call it when the node \xmlNode{Code} is found parsing the XML input
file.
%
The method receives from RAVEN an attribute of type ``xml.etree.ElementTree'',
containing all the sub-nodes and attribute of the XML block \xmlNode{Code}.
%

Example XML:
\begin{lstlisting}[style=XML,morekeywords={subType,ModuleToLoad}]
<Models>
  ...
  <Code name='newCode' subType='NewPluginName.NewCodePluginEntity'>
    <!--
      here we define other XML nodes RAVEN does not read automatically.
      We need to implement, in the code Plugin class the _readMoreXML
      method
    -->
    <newNodeWeNeedToRead>
      whatNeedsToBeRead
    </newNodeWeNeedToRead>
  </ExternalModel>
  ...
</Models>
\end{lstlisting}
