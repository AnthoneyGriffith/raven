\section{Making a New RAVEN Plugin}

Creating a new plugin is a straightforward process. It involves setting up a repository,
establishing a basic structure, and installing in RAVEN for testing.

\subsection{Setting up a Repository}
TODO

\subsection{Plugin Structure}
The following directories must be present in the main directory of the plugin in order for RAVEN to
read it correctly:
\begin{itemize}
  \item \texttt{src}, where the entities for RAVEN to load are located;
  \item \texttt{doc}, where the documentation for the plugin and its entities is located.
  \item \texttt{tests}, where continuous integration tests are located;
\end{itemize}

\subsection{Additional Libraries}
If the plugin requires additional libraries, they can extend the \texttt{dependencies.ini} file in
the same manner as RAVEN's dependencies file. Libraries will be added like they are for RAVEN
itself, and a check will be performed to assure no base RAVEN (or other plugin) dependencies are
modified.

\subsection{Installing in RAVEN}
Use the installation script in \texttt{raven/install\_plugins.py}.

This process automatically registers the plugin in the plugin directory, and informs the plugin
about RAVEN (TODO more notes on this!)