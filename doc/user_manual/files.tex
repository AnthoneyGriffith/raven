\section{Files}
\label{sec:files}

The \xmlNode{Files} block defines any files that might be needed within
the RAVEN run.  This could include inputs to the Model, pickled ROM files,
or CSV files for postprocessors, to name a few.
%
Each entry in the \xmlNode{Files} block is a tag with the file type.  Files
given through the input XML at this point are all \xmlNode{Input} type.
Each \xmlNode{Input} node has the following attributes:
\vspace{-5mm}
\begin{itemize}
  \itemsep0em
  \item \xmlAttr{name}, \xmlDesc{required string attribute}, user-defined name
  of the file.  This does not need to be the actual filename; this is the name
  by which RAVEN will identify the file.
  %
  \nb As with other objects, this name can be used to refer to this
  specific entity from other input blocks in the XML.
  %
  \item \xmlAttr{type}, \xmlDesc{optional string attribute}, a type label
  for this file.  While RAVEN does not directly make use of file types,
  they are available in the CodeInterface as identifiers.  If not provided,
  the type will be stored as python \texttt{None} type.
  %

  \item \xmlAttr{perturbable}, \xmlDesc{optional boolean attribute}, flag
  to indicate whether a file can be perturbed or not. RAVEN does not
  directly use this attribute, but it is availabe in the CodeInterface.
  If not provided, defaults to True.
  %
  
  \item \xmlAttr{subDirectory}, \xmlDesc{optional string attribute}, sub-directory
  that should be created in the perturbation process. The file specified in the body of the 
  XML node should be located in the  \xmlAttr{subDirectory} under the  \xmlAttr{workingDir} specified
  in the \xmlNode{RunInfo} XML block (i.e. $workingDir/subDirectory$).
  If specified, the file will be
  placed in the sub-directory. For example, in a \textit{MultiRun} step, 
  the file will be copied into \\$workingDir/stepName/\%counter\%/subDirectory$, where $workingDir$
  is the working directory specified in the \textit{RunInfo} XML block, $stepName$ is the name of the step, 
  $/\%counter\%$ is the realization identifier (e.g. $1$,$2$, etc.) and $subDirectory$ is the sub-directory here
  specified.
  If not provided, defaults to an empty string.
  %
\end{itemize}
\vspace{-5mm}
For example, if the files \texttt{templateInput.i, materials.i, history.i, mesh.e}
 are required to run a Model, the \xmlNode{Files} block might appear as:
\begin{lstlisting}[style=XML,morekeywords={name,file}] %moreemph={name,file}]
<Simulation>
  ...
  <Files>
    <Input name='main' type='maininput'>templateInput.i</Input>
    <Input name='mat'  type='mtlinput' >materials.i</Input>
    <Input name='hist' type='histinput'>history.i</Input>
    <Input name='mesh' type='mesh' perturbable='false'>mesh.e</Input>
    <Input name='fileInSubDir' type='' subDirectory="theSubDirectory">theFileInTheSubDir.inp</Input>
  </Files>
   ...
</Simulation>
\end{lstlisting}
