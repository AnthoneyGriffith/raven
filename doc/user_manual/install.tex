\section{Installation Overview}

The installation of the RAVEN code is a straightforward procedure;
depending on the usage purpose and machine architecture, the
installation process slightly differs.

In the following sections, all the different installation procedures
are reported.

There are two main requirements to installing RAVEN, installing the
RAVEN dependencies (Section \ref{raven_dependencies}) and installing
RAVEN itself (Section \ref{raven_installation}).  For any particular
installation, only one of the raven dependency procedures and one of
the raven installation paths needs to be taken.

For macOS (OSX) it is recommended that the dependencies be installed with
Miniconda (Section \ref{miniconda}).  For Linux, it is recommended
that the distribution package manager be used if possible (Section
\ref{sysprep_linux}).

There are several different ways of getting RAVEN as described in
Section \ref{raven_installation}.  They vary depending on how easy
they are to use and how easy they are to develop with (these tend to
be inversely correlated).

\newcommand{\goToRavenInstallation}{Now go on to Section \ref{raven_installation} for Raven installation.
}


\section{RAVEN Dependencies Installation}
\label{raven_dependencies}

RAVEN is built upon several freely available open-source software packages,
which must  be installed before it will function properly:

\begin{center}
    \begin{tabular}{ | l | p{10cm} |}
    \hline Package & Purpose \\
    \hline git & Source code control tool \\
    \hline g++ & C++ language compiler suite (Needed to build
        support code in the RAVEN package) \\
    \hline libtool & Generic library management tool \\
    \hline python & Scripting language RAVEN is written in \\
    \hline swig & Simplified Wrapper and Interface Generator
        (Used to create Python interfaces to supporting C++ code) \\
    \hline hdf5 & Library providing interface to HDF5 database
        file format used by RAVEN \\
    \hline h5py & Python interface to HDF5 database library \\
    \hline numpy & N-Dimensional array package for Python \\
    \hline scipy &  Scientific computing package for Python \\
    \hline scikit-learn & Machine learning library for Python \\
    \hline matplotlib & Plotting library for Python \\
    \hline
    \end{tabular}
\end{center}


RAVEN is supported on three separate computing platforms:
Linux, OSX (Apple Macintosh), and Microsoft Windows.  Depending
on which of these systems is used, the preparation of the system
to run RAVEN varies.

\subsection{Preparing a Linux System for RAVEN}
\label{sysprep_linux}

When installing RAVEN on a Linux system, it is recommended that the
distribution's package manager be used to install the dependencies.
Doing so automates the process and automatically includes any needed
dependencies of the requested packages.  Below are instructions for
doing so for two popular Linux distributions, Ubuntu and Fedora.  If
the distribution's packages are not available or too old, either a
direct Miniconda (See Section \ref{miniconda_direct}) or a manual
dependency install (See Section \ref{sysprep_manual}) can be used.

\subsubsection{Ubuntu}

The Ubuntu distribution of Linux makes use of the Advanced Package
Tool (APT) to automate the installation of preconfigured software.
Ubuntu 16.4 or newer provide all the packages needed for RAVEN.  The
following command uses the APT to add the needed packages and any
needed dependencies not already on the system:

\begin{lstlisting}[language=bash]
sudo apt-get install libtool git python-dev swig g++ \
  python3-dev python-numpy python-sklearn python-h5py
\end{lstlisting}

Optionally, if you want to be able to edit and rebuild the manual, you can
install \TeX~Live and its related packages:
\begin{lstlisting}[language=bash]
sudo apt-get install texlive-latex-base texlive-extra-utils \
  texlive-latex-extra texlive-math-extra
\end{lstlisting}

\goToRavenInstallation

\subsubsection{Fedora}

When using the Fedora Linux distribution, use the RPM package manager to
install needed dependencies:

\begin{lstlisting}[language=bash]
dnf install swig libtool gcc-c++ redhat-rpm-config python-devel \
  python3-devel numpy h5py scipy python-scikit-learn \
  python-matplotlib-qt4
\end{lstlisting}

In addition, if you would like to be able to build the documentation
included in the RAVEN software distribution it is necessary to install
\TeX~Live and its related packages:
\begin{lstlisting}[language=bash]
dnf install texlive texlive-subfigure texlive-stmaryrd \
  texlive-titlesec texlive-preprint texlive-placeins \
  texlive-bigints texlive-relsize texlive-appendix
\end{lstlisting}

Note: The 'dnf' command replaces 'yum' used on older versions of
Fedora Linux.

\goToRavenInstallation

\subsection{Preparing an Apple Macintosh OSX System for RAVEN}
\label{sysprep_osx}

When using an Apple Macintosh computer, the above dependencies are met
by following three steps: Installing the XCode command line tools from Apple,
installing the XQuartz  X-Window system server, and then installing and using the Miniconda
package system to add the rest of the dependencies.

\subsubsection{Installing XCode Command Line Tools}

The XCode command line tools package from Apple Computer provides the C++
compilers and git source code control tools needed to obtain and build RAVEN.
It is freely available from the Apple developer site (download and install the appropriate link
for your version of OSX):
\begin{itemize}
    \item For El Capitan download and install: \url{https://developer.apple.com/downloads/index.action?name=for%20Xcode}
    \item For Yosemite download and install:  \url{https://developer.apple.com/downloads/index.action?name=for%20Xcode%20-}
\end{itemize}

\subsubsection{Installing XQuartz}
XQuartz is an implementation of the X Server for the Mac OSX operating system.  It is required
becuase RAVEN is often used with applications developed with the Multiphysics
Object-Oriented Simulation Environment (MOOSE).
\begin{itemize}
    \item Download and install: \url{https://dl.bintray.com/xquartz/downloads/XQuartz-2.7.9.dmg}
\end{itemize}

\subsubsection{Installing Miniconda and Packages}
\label{miniconda}

For OSX, the simplest and most robust way to install the RAVEN
dependencies is with Miniconda.  Miniconda is a package manager for
python packages and can be used to install the third party packages
that RAVEN depends on.

For OSX Yosemite and OSX El Capitan there is a package called
raven\_miniconda.pkg available.  The package includes both miniconda
and the dependencies RAVEN needs. If you get an error that the package
is not signed, then Control click the package, and choose ``Open
With'' and then Installer.  The file can be gotten from the RAVEN wiki
from a link on the page:
\url{https://hpcgitlab.inl.gov/idaholab/raven/wikis/installationMAC}

The files will be installed into \texttt{/opt/raven\_libs}.  RAVEN
will look at this directory during both compiling and running to get
the RAVEN libraries from.

%% Your \texttt{.bash\_profile} will be modified to source
%% the\\ \texttt{/opt/raven\_libs/environments/raven\_libs\_profile}
%% file.  This needs to be sourced either with the command:
%% \begin{lstlisting}[language=bash]
%% source $HOME/.bash_profile
%% \end{lstlisting}
%% or by reopening the terminal will put the correct Miniconda python in
%% the path.

Once this is installed you can go on to Section \ref{raven_installation} and install RAVEN.

Advanced note: This creates a conda environment called
\texttt{raven\_libraries} and the command used to create this the \texttt{raven\_libraries} environment can be found by by running:
\texttt{python scripts/TestHarness/testers/RavenUtils.py --conda-create}

\label{miniconda_direct}

Alternatively, Miniconda can be obtained and installed directly. See
\url{http://conda.pydata.org/miniconda.html} for more information and
installation files to download.  If your system already has Miniconda,
you may use that installation to load these packages.

After Miniconda has been installed, it may then be used to install the
rest of the packages needed to run RAVEN (this command also creates a
separate environment for the RAVEN libraries):

\lstinputlisting[language=bash]{conda_command.txt}

\subsection{Preparing a Windows System for RAVEN}
\label{sysprep_windows}

Since RAVEN requires a UNIX-like shell to function, one must be installed for a Microsoft
Windows system to run it.  A freely available software package called MSYS2 is used to
provide this functionality.  More information about MSYS2 is available at
\url{https://sourceforge.net/p/msys2/wiki/MSYS2%20introduction/}.

\subsubsection{Prerequisites to use RAVEN on Windows}
\begin{itemize}
    \item A system running a 64-bit version of Microsoft Windows. Installation and operation
        has been verified on Windows 7 and Windows Server 2012 R2 Standard. While there
        is also a 32-bit version of MSYS2 available, the RAVEN installation described here will not work with it.
    \item At least 7 Gigabytes of available disk space:
    \begin{itemize}
        \item 3.5 GB for MSYS2, including everything necessary to build and run RAVEN: Python, GNU compilers, required Python packages, git
        \item 1.5 GB for RAVEN
        \item 1.1 GB for the MOOSE framework, upon which RAVEN is based
    \end{itemize}
\end{itemize}

\subsubsection{Installation}
\begin{enumerate}
    \item Obtain and run the latest basic 64-bit MSYS2 installer from \url{ https://msys2.github.io/} (As of this writing it is named
	msys2-x86\_64-20160205.exe and is approximately 50 Megabytes in size).
    \item The page with the download also contains installation instructions. Perform the steps described there up to
	step 4 to install a minimal MSYS2 system. Make sure that you install to path C:\textbackslash{}msys64. There
	is no need to go past step 4 because the balance of what is needed to run RAVEN under MSYS2 have been
	pre-packaged together in one large archive file.
    \item Download the pre-built archive file containing MSYS2 packages needed to run RAVEN.  A link to the current version of
	this file may be found in the Windows installation section of the RAVEN Wiki:
	\url{https://hpcgitlab.inl.gov/idaholab/raven/wikis/installationWindows#installation}
    \item Use a Windows archive file program such as WinZip to open the large zip file just downloaded.  Shut down any
	MSYS2 shells you have open and then copy the "msys64" folder from the large zip file so that it is copied right
	on top of the MSYS2 installation performed above. When the file copying is done (which will take a while)
	you should have an environment ready to receive RAVEN.
    \item This installation will create shortcuts that may be used to start UNIX-Like shells:
	\begin{itemize}
	    \item MSYS2 Shell
	    \item MinGW-w64 Win32 Shell
	    \item MinGW-w64 Win64 Shell
	\end{itemize}
	For best results with Python, a copy of the shortcut pointing to  "MinGW-w64 Win64 Shell" needs to be created. Give
	it a name like " MinGW-w64 Win64 SH".  Open the properties of this new shortcut an add "-sh" to the end of the Target
	field.  Use this new shortcut to start shells that will be used to run RAVEN.
\end{enumerate}


\subsubsection{For More Information}
Note: An illustrated version of this procedure may be found on the Wiki page at
\url{https://hpcgitlab.inl.gov/idaholab/raven/wikis/installationWindows}.

\subsection{Manual Dependency Install}
\label{sysprep_manual}

If other options don't work, the dependencies can be installed
manually.  This is sometimes tricky, pay attention to the order you
install them.  RAVEN uses the following packages (newer versions
usually work):

\begin{enumerate}
  \input{libraries.tex}
\item hdf5 1.8 or newer
\item swig 2.0 or newer
\item gcc 4.9 or newer (or other C++11 compiler)
\item TeX Live 2016 or newer (to build manuals)
\end{enumerate}

\goToRavenInstallation

\subsubsection{PIP Install}

PIP (Pip Installs Packages) is a package management system that can be
used to install Python packages.  It cannot be used to install swig or
hdf5, so those may need to be manually installed.  In general, PIP
cannot be used with MSYS on Windows.  Using a virtual environment with
PIP is recommended (without the virtual environment, only the
\texttt{pip install} command is needed):

\lstinputlisting[language=bash]{pip_commands.txt}

Note that the \texttt{source} command will need to be run to use the
environment in the future when the shell is restarted since the
virtual environment is only available after switching to it.

The following command can be used to find out the pip install command
for the RAVEN dependencies:

\texttt{python scripts/TestHarness/testers/RavenUtils.py --pip-install}

\goToRavenInstallation


\section{RAVEN Installation}
\label{raven_installation}

Once the RAVEN dependencies have been installed (See Section
\ref{raven_dependencies}), the rest of RAVEN can be installed.

There are two different ways to get RAVEN.  There are trade offs
between how easy it is to setup and how easy it is to develop with the
method.  If you want to do development on RAVEN or keep up with RAVEN
changes go to Section \ref{submodule_git}, which describes obtaining
the software directly from the online repository.  Otherwise, RAVEN
may be installed from a stand alone source code package (Section
\ref{raven_source_package}).


\subsection{Submodule Git}
\label{submodule_git}

This install method uses git to obtain the software and uses
submodules to get MOOSE.  This can be used for RAVEN development.

First RAVEN needs to be cloned, and then the submodules initialized.

\begin{lstlisting}[language=bash]
git clone https://github.com/idaholab/raven.git
cd raven
git submodule update --init moose
\end{lstlisting}

Next follow the compilation and testing instructions in Section \ref{raven_compilation}.

To update the software, the git pull and submodule update commands can
be used:

\begin{lstlisting}[language=bash]
git pull
git submodule update
\end{lstlisting}


\subsection{RAVEN Source Code Package}
\label{raven_source_package}

Untar the source install (if there is more than one version of the
source tarball, the full filename will need to be used instead of *):

\begin{lstlisting}[language=bash]
tar -xvzf raven_framework_*_source.tar.gz
cd raven
\end{lstlisting}

Next follow the compilation and testing instructions in Section
\ref{raven_compilation}.

\subsection{RAVEN Compilation}
\label{raven_compilation}

The RAVEN modules should be compiled:

\begin{lstlisting}[language=bash]
#change into the raven directory if needed.
make framework_modules
\end{lstlisting}

Then the testing should be done:

\begin{lstlisting}[language=bash]
./run_framework_tests
\end{lstlisting}

The output should describe why any tests failed.

At the end, there should be a line that looks similar to the output below:
\begin{lstlisting}[language=bash]
8 passed, 19 skipped, 0 pending, 0 failed
\end{lstlisting}

Normally there are skipped tests because either some of the codes are
not available, or some of the test are not currently working.  The
output will explain why each is skipped.

If all the tests pass, you are ready to read about Running RAVEN in
Section \ref{HowToRun}.

If the tests did not pass, check Section
\ref{troubleshooting_installation} on troubleshooting.

\subsection{Troubleshooting the Installation}
\label{troubleshooting_installation}

Often the problems result from one or more of the libraries being
incorrect or missing.  In the raven directory, the command:

\begin{lstlisting}[language=bash]
./run_tests --library_report
\end{lstlisting}
can be used to check if all the libraries are available, and which
ones are being used.  If amsc, distribution1D or interpolationND are
missing, then the RAVEN modules need to be compiled or recompiled.
Otherwise, the RAVEN dependencies need to be fixed.

Note, that when using RAVEN remotely in a graphical session with X11
forwarded to the client, some tests may depend on live X11 forwarding
to the remote client. If the user is not using X11 forwarding then
RAVEN will work fine and not use X11.  However, when the user has
forwarded their X11 environment to a ssh client, the connection may
timeout. The standard timeout of X for an untrusted connection is 20
minutes.  The full test suite including those involving graphical
output can take longer than the aforementioned timeout. One way to
alleviate this is to login to the remote host of RAVEN using a trusted
connection by using the -Y flag:

\begin{lstlisting}[language=bash]
ssh -Y hostname
\end{lstlisting}

This should only be done when the user is using a secure connection to
a known host though, as there are security concerns to the client
machine when allowing a remote computer access to its graphical user
interface.


\subsection{In-use Testing}

In use testing can be done by re-running the installation tests as
described in Section \ref{raven_compilation}.
