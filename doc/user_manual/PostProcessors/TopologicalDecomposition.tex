\subsubsection{TopologicalDecomposition}
\label{TopologicalDecomposition}
The \textbf{TopologicalDecomposition} post-processor will compute an
approximated hierarchical Morse-Smale complex which will add two columns to a
dataset, namely \texttt{minLabel} and \texttt{maxLabel} that can be used to
decompose a dataset.
%

The topological post-processor can also be run in `interactive' mode, that is
by passing the keyword \texttt{interactive} to the command line of RAVEN's
driver.
%
In this way, RAVEN will initiate an interactive UI that allows one to explore
the topological hierarchy in real-time and adjust the simplification setting
before adjusting a dataset. Use in interactive mode will replace the parameter
\xmlNode{simplification} described below with whatever setting is set in the UI
upon exiting it.

In order to use the \textbf{TopologicalDecomposition} post-processor, the user
needs to set the attribute \xmlAttr{subType}:
\xmlNode{PostProcessor \xmlAttr{subType}=\xmlString{TopologicalDecomposition}}.
The following is a list of acceptable sub-nodes:
\begin{itemize}
  \item \xmlNode{graph} \xmlDesc{, string, optional field}, specifies the type
  of neighborhood graph used in the algorithm, available options are:
  \begin{itemize}
    \item \texttt{beta skeleton}
    \item \texttt{relaxed beta skeleton}
    \item \texttt{approximate knn}
    %\item Delaunay \textit{(disabled)}
  \end{itemize}
  \default{\texttt{beta skeleton}}
  \item \xmlNode{gradient}, \xmlDesc{string, optional field}, specifies the
  method used for estimating the gradient, available options are:
  \begin{itemize}
    \item \texttt{steepest}
    %\item \xmlString{maxflow} \textit{(disabled)}
  \end{itemize}
  \default{\texttt{steepest}}
  \item \xmlNode{beta}, \xmlDesc{float in the range: (0,2], optional field}, is
  only used when the \xmlNode{graph} is set to \texttt{beta skeleton} or
  \texttt{relaxed beta skeleton}.
  \default{1.0}
  \item \xmlNode{knn}, \xmlDesc{integer, optional field}, is the number of
  neighbors when using the \xmlString{approximate knn} for the \xmlNode{graph}
  sub-node and used to speed up the computation of other graphs by using the
  approximate knn graph as a starting point for pruning. -1 means use a fully
  connected graph.
  \default{-1}
  \item \xmlNode{weighted}, \xmlDesc{boolean, optional}, a flag that specifies
  whether the regression models should be probability weighted.
  \default{False}
  \item \xmlNode{interactive}, if this node is present \emph{and} the user has
  specified the keyword \texttt{interactive} at the command line, then this will
  initiate a graphical interface for exploring the different simplification
  levels of the topological hierarchy. Upon exit of the graphical interface, the
  specified simplification level will be updated to use the last value of the
  graphical interface before writing any ``output'' results.
  \item \xmlNode{persistence}, \xmlDesc{string, optional field}, specifies how
  to define the hierarchical simplification by assigning a value to each local
  minimum and maximum according to the one of the strategy options below:
  \begin{itemize}
    \item \texttt{difference} - The function value difference between the
    extremum and its closest-valued neighboring saddle.
    \item \texttt{probability} - The probability integral computed as the
    sum of the probability of each point in a cluster divided by the count of
    the cluster.
    \item \texttt{count} - The count of points that flow to or from the
    extremum.
    % \item \xmlString{area} - The area enclosed by the manifold that flows to
    % or from the extremum.
  \end{itemize}
  \default{\texttt{difference}}
  \item \xmlNode{simplification}, \xmlDesc{float, optional field}, specifies the
  amount of noise reduction to apply before returning labels.
  \default{0}
  \item \xmlNode{parameters}, \xmlDesc{comma separated string, required field},
  lists the parameters defining the input space.
  \item \xmlNode{response}, \xmlDesc{string, required field}, is a single
  variable name defining the scalar output space.
\end{itemize}
\textbf{Example:}
\begin{lstlisting}[style=XML,morekeywords={subType}]
<Simulation>
  ...
  <Models>
    ...
    <PostProcessor name="***" subType='TopologicalDecomposition'>
      <graph>beta skeleton</graph>
      <gradient>steepest</gradient>
      <beta>1</beta>
      <knn>8</knn>
      <normalization>None</normalization>
      <parameters>X,Y</parameters>
      <response>Z</response>
      <weighted>true</weighted>
      <simplification>0.3</simplification>
      <persistence>difference</persistence>
    </PostProcessor>
    ...
  <Models>
  ...
<Simulation>
\end{lstlisting}
