\section{Steps}
\label{sec:steps}
The core of the RAVEN calculation flow is the \textbf{Step} system.
%
The \textbf{Step} is in charge of assembling different entities in RAVEN (e.g.
Samplers, Models, Databases, etc.) in order to perform a task defined by the
kind of step being used.
%
A sequence of different \textbf{Steps} represents the calculation flow.
%

Before analyzing each \textbf{Step} type, it is worth to 1)
explain how a general \textbf{Step} entity is organized, and 2) introduce the concept of step
``role'' .
%
In the following example, a general example of a \textbf{Step} is shown below:
\begin{lstlisting}[style=XML,morekeywords={class}]
<Simulation>
  ...
  <Steps>
    ...
    <WhatEverStepType name='aName'>
        <Role1 class='aMainClassType' type='aSubType'>userDefinedName1</Role1>
        <Role2 class='aMainClassType' type='aSubType'>userDefinedName2</Role2>
        <Role3 class='aMainClassType' type='aSubType'>userDefinedName3</Role3>
        <Role4 class='aMainClassType' type='aSubType'>userDefinedName4</Role4>
    </WhatEverStepType>
    ...
  </Steps>
  ...
</Simulation>
\end{lstlisting}
As shown above each \textbf{Step} consists of
a list of entities organized into ``Roles.''
%
Each role represents a behavior the entity (object) will assume during the
evaluation of the \textbf{Step}.
%
In RAVEN, several different roles are available:
\begin{itemize}
\item \textbf{Input} represents the input of the \textbf{Step}.
The allowable input objects depend on the type of \textbf{Model} in the
\textbf{Step}.
\item \textbf{Output} defines where to collect the results of an action
performed by the \textbf{Model}.
It is generally one of the following types: \textbf{DataObjects}, \textbf{Databases},
or \textbf{OutStreams}.
\item \textbf{Model} represents a physical or mathematical system or behavior.
The object used in this role defines the allowable types of
\textbf{Inputs} and \textbf{Outputs} usable in this step.
\item \textbf{Sampler} defines the sampling strategy to be used to probe the model.
\\ It is worth to mention that, when a sampling strategy is employed, the ``variables'' defined in the \xmlNode{variable} blocks are going to be
directly placed in the \textbf{Output} objects of type \textbf{DataObjects} and \textbf{Databases}).
\item \textbf{Function} is an extremely important role. It introduces the capability to
perform pre or post processing of Model \textbf{Inputs} and \textbf{Outputs}. Its specific
behavior depends on the \textbf{Step} is using it.
%\item \textbf{Function.} The Function role is extremely important, for example, when performing Adaptive Sampling to represent the metric of the transition regions. This role is the role used, for example, to collapse information coming from a Model.
\item \textbf{ROM} defines an acceleration Reduced Order Model to use for a
\textbf{Step}.
\item \textbf{SolutionExport} the DataObject to store solutions from Optimizer or Sampler execution
in a Step. For example, a LimitSurfaceSearch Sampler outputs the coordinates of the limit surface; 
similarly, an Optimizer outputs the convergence history and optimal points. 
See specific Samplers and Optimizers for details.
\end{itemize}
Depending on the \textbf{Step} type, different combinations of these roles can
be used.
For this reason, it is important to analyze each \textbf{Step} type in details.

The available steps are the following
\begin{itemize}
\item SingleRun (see Section~\ref{subsec:stepSingleRun})
\item MultiRun(see Section~\ref{subsec:stepMultiRun})
\item IOStep(see Section~\ref{subsec:stepIOStep})
\item RomTrainer(see Section~\ref{subsec:stepRomTrainer})
\item PostProcess(see Section~\ref{subsec:stepPostProcess})
\end{itemize}


%%%%%%%%%%%%%%%%%%%%
 %%%%% SINGLERUN %%%%%
%%%%%%%%%%%%%%%%%%%%
\subsection{SingleRun}
\label{subsec:stepSingleRun}
The \textbf{SingleRun} is the simplest step the user can use to assemble a
calculation flow: perform a single action of a \textbf{Model}.
%
For example, it can be used to run a single job (Code Model) and collect the
outcome(s) in a ``\textbf{DataObjects}'' object of type \textbf{Point} or
\textbf{History} (see Section~\ref{sec:DataObjects} for more details on available data
representations).

The specifications of this Step must be defined within a \xmlNode{SingleRun} XML
block.
%
This XML node has the following definable attributes:
\vspace{-5mm}
\begin{itemize}
\itemsep0em
\item \xmlAttr{name}, \xmlDesc{required string attribute}, user-defined name of
  this \textbf{Step}. \nb This name is used to reference this specific entity
  in the \xmlNode{RunInfo} block, under the \xmlNode{Sequence} node. If the name
  of this \textbf{Step} is not listed in the  \xmlNode{Sequence} block, its action is not
  going to be performed.
\item \xmlAttr{repeatFailureRuns}, \xmlDesc{optional integer attribute}, this optional
  attribute could be used to set a certain number of repetitions that need to be performed
  when a realization (i.e. run) fails (e.g. \xmlAttr{repeatFailureRuns} = ``3'', 3 tries).
\item \xmlAttr{pauseAtEnd}, \xmlDesc{optional boolean/string attribute (case insensitive)}, if True
  (True values = True, yes, y, t), the code will pause at the end of
  the step, waiting for a user signal to continue. This is used in case one or
  more of the \textbf{Outputs} are of type \textbf{OutStreams}.
  For example, it can be used when an \textbf{OutStreams} of type
  \textbf{Plot} is output to the screen. Thus, allowing the user to interact with
  the \textbf{Plot} (e.g. rotate the figure, change the scale, etc.).
  \default{False}.
\item \xmlAttr{clearRunDir}, \xmlDesc{optional boolean attribute}, indicates whether the run
  directory should be cleared (removed) before beginning
  the Step calculation. The run directory has the same \xmlAttr{name} as the \xmlNode{Step} and is
  located within the \xmlNode{WorkingDir}. Note this directory is only used for a \xmlNode{Step}
  with certain \xmlNode{Model} types, such as \xmlNode{Code}.
\end{itemize}

In the \xmlNode{SingleRun} input block, the user needs to specify the objects
needed for the different allowable roles.
%
This step accepts the following roles:
\begin{itemize}
\item \xmlNode{Input}, \xmlDesc{string, required parameter}, names an entity
(defined elsewhere in the RAVEN input) that will be used as input for the model
specified in this step.
This XML node accepts the following attributes:
\begin{itemize}
  \item \xmlAttr{class}, \xmlDesc{required string attribute}, main object class
    type.
    This string corresponds to the tag of the main object's type used in the
    input.
    For example, \xmlString{Files}, \xmlString{DataObjects}, \xmlString{Databases},
    etc.
  \item \xmlAttr{type}, \xmlDesc{required string attribute}, the actual entity
    type.
    This attribute needs to specify the object type within the main object
    class.
    For example, if the  \xmlAttr{class} attribute is \xmlString{DataObjects}, the
    \xmlAttr{type} attribute might be \xmlString{PointSet}.
    \nb The class \xmlString{Files} has no type (i.e.
    \xmlAttr{type}\textbf{\texttt{=''}}).
\end{itemize}

\nb The \xmlAttr{class} and, consequently, the \xmlAttr{type} usable for this
role depends on the particular \xmlNode{Model} being used.
%
In addition, the user can specify as many \xmlNode{Input} nodes as needed.

\item \xmlNode{Model}, \xmlDesc{string, required parameter}, names an entity
defined elsewhere in the input file to be used as a model for this step.
This XML node accepts the following attributes:
\begin{itemize}
  \item \xmlAttr{class}, \xmlDesc{required string attribute}, main object class
    type.
    For this role, only \xmlString{Models} can be used.
  \item \xmlAttr{type}, \xmlDesc{required string attribute}, the actual entity
    type.
    This attribute needs to specify the object type within the \texttt{Models}
    object class.
    For example, the \xmlAttr{type} attribute might be \xmlString{Code},
    \xmlString{ROM}, etc.
\end{itemize}
\item \xmlNode{Output}, \xmlDesc{string, required parameter} names an entity
defined elsewhere in the input to use as the output for the \textbf{Model}.
This XML node recognizes the following attributes:
\begin{itemize}
  \item \xmlAttr{class}, \xmlDesc{required string attribute}, main object class
    type.
    For this role, only \xmlString{DataObjects}, \xmlString{Databases}, and
    \xmlString{OutStreams} can be used.
  \item \xmlAttr{type}, \xmlDesc{required string attribute}, the actual entity
    type.
    This attribute needs to specify the object type within the main object
    class.
    For example, if the  \xmlAttr{class} attribute is \xmlString{DataObjects}, the
    \xmlAttr{type} attribute might be \xmlString{PointSet}.
\end{itemize}
\nb The number of \xmlNode{Output} nodes is unlimited.
\end{itemize}

Example:
\begin{lstlisting}[style=XML,morekeywords={class,pauseAtEnd}]
<Steps>
  ...
  <SingleRun name='StepName' pauseAtEnd='false'>
        <Input    class='Files'          type=''>anInputFile.i</Input>
        <Input    class='Files'          type=''>aFile</Input>
        <Model   class='Models'      type='Code'>aCode</Model>
        <Output  class='Databases' type='HDF5'>aDatabase</Output>
        <Output  class='DataObjects'        type='History'>aData</Output>
  </SingleRun>
  ...
</Steps>
\end{lstlisting}
%%%%%%%%%%%%%%%%%%%
 %%%%% MULTIRUN %%%%%
%%%%%%%%%%%%%%%%%%%
\subsection{MultiRun}
\label{subsec:stepMultiRun}
The \textbf{MultiRun} step allows the user to assemble the calculation flow of
an analysis that requires multiple ``runs'' of the same model.
%
This step is used, for example, when the input (space) of the model needs to be
perturbed by a particular sampling strategy.
%

The specifications of this type of step must be defined within a
\xmlNode{MultiRun} XML block.
%
This XML node recognizes the following list of attributes:
\vspace{-5mm}
\begin{itemize}
\itemsep0em
\item \xmlAttr{name}, \xmlDesc{required string attribute}, user-defined name of
this Step.
\nb As with other objects, this name is used to reference this specific entity
in the \xmlNode{RunInfo} block, under the \xmlNode{Sequence} node. If the name
of this \textbf{Step} is not listed in the  \xmlNode{Sequence} block, its action is not
going to be performed.
\item \xmlAttr{re-seeding}, \xmlDesc{optional integer/string attribute}, this optional
attribute could be used to control the seeding of the random number generator (RNG).
If inputted, the RNG can be reseeded. The value of this attribute
can be: either 1) an integer value with the seed to be used (e.g. \xmlAttr{re-seeding} =
``20021986''), or 2) string value named ``continue'' where the RNG is not re-initialized
\item \xmlAttr{repeatFailureRuns}, \xmlDesc{optional integer attribute}, this optional
attribute could be used to set a certain number of repetitions that need to be performed
when a realization (i.e. run) fails (e.g. \xmlAttr{repeatFailureRuns} = ``3'', 3 tries).

\item \xmlAttr{pauseAtEnd}, \xmlDesc{optional boolean/string attribute}, if True
(True values = True, yes, y, t), the code will pause at the end of
the step, waiting for a user signal to continue. This is used in case one or
more of the \textbf{Outputs} are of type \textbf{OutStreams}.
For example, it can be used when an \textbf{OutStreams} of type
\textbf{Plot} is output to the screen. Thus, allowing the user to interact with
the \textbf{Plot} (e.g. rotate the figure, change the scale, etc.).
\item \xmlAttr{sleepTime}, \xmlDesc{optional float attribute}, in this attribute
the user can specify the waiting time (seconds) between two subsequent inquiries
of the status of the submitted job (i.e. check if a run has finished).
\default{0.05}.
\end{itemize}
\vspace{-5mm}
In the \xmlNode{MultiRun} input block, the user needs to specify the objects
that need to be used for the different allowable roles.
%
This step accepts the following roles:
\vspace{-5mm}
\begin{itemize}
\item \xmlNode{Input}, \xmlDesc{string, required parameter}, names an entity to
be used as input for the model specified in this step.
This XML node accepts the following attributes:
\begin{itemize}
  \item \xmlAttr{class}, \xmlDesc{required string attribute}, main object class
    type.
    This string corresponds to the tag of the main object's type used in the
    input.
    For example, \xmlString{Files}, \xmlString{DataObjects}, \xmlString{Databases},
    etc.
  \item \xmlAttr{type}, \xmlDesc{required string attribute}, the actual entity
    type.
    This attribute specifies the object type within the main object class.
    For example, if the  \xmlAttr{class} attribute is \xmlString{DataObjects}, the
    \xmlAttr{type} attribute might be \xmlString{PointSet}.
    \nb The class \xmlString{Files} has no type (i.e.
    \xmlAttr{type}\textbf{\texttt{=''}}).
\end{itemize}
\nb The \xmlAttr{class} and, consequently, the \xmlAttr{type} usable for this
role depend on the particular \xmlNode{Model} being used.
The user can specify as many \xmlNode{Input} nodes as needed.
\item \xmlNode{Model}, \xmlDesc{string, required parameter} names an entity
defined elsewhere in the input that will be used as the model for this step.
This XML node recognizes the following attributes:
\begin{itemize}
  \item \xmlAttr{class}, \xmlDesc{required string attribute}, main object class
    type.
    For this role, only \xmlString{Models} can be used.
  \item \xmlAttr{type}, \xmlDesc{required string attribute}, the actual entity
    type.
    This attribute needs to specify the object type within the \texttt{Models}
    object class.
    For example, the \xmlAttr{type} attribute might be \xmlString{Code},
    \xmlString{ROM}, etc.
\end{itemize}
\item \xmlNode{Sampler}, \xmlDesc{string, optional parameter} names an entity
defined elsewhere in the input file to be used as a sampler.
As mentioned in Section \ref{sec:Samplers}, the \textbf{Sampler} is in charge of
defining the strategy to characterize the input space.
This XML node recognizes the following attributes:
\begin{itemize}
  \item \xmlAttr{class}, \xmlDesc{required string attribute}, main object class
    type.
    This string corresponds to the tag of the main object's type used.
    Only \xmlString{Samplers} can be used for this role.
  \item \xmlAttr{type}, \xmlDesc{required string attribute}, the actual entity
    type.
    This attribute needs to specify the object type within the \texttt{Samplers}
    object class.
    For example, the \xmlAttr{type} attribute might be \xmlString{MonteCarlo},
   \xmlString{Adaptive}, \xmlString{AdaptiveDET}, etc.
    See Section \ref{sec:Samplers} for all the different types currently
    supported.
\end{itemize}
\item \xmlNode{Optimizer}, \xmlDesc{string, optional parameter} names an entity
defined elsewhere in the input file to be used as an optimizer.
As mentioned in Section \ref{sec:Optimizers}, the \textbf{Optimizer} is in charge of
defining the strategy to optimize an user-specified variable.
This XML node recognizes the following attributes:
\begin{itemize}
  \item \xmlAttr{class}, \xmlDesc{required string attribute}, main object class
    type.
    This string corresponds to the tag of the main object's type used.
    Only \xmlString{Optimizers} can be used for this role.
  \item \xmlAttr{type}, \xmlDesc{required string attribute}, the actual entity
    type.
    This attribute needs to specify the object type within the \texttt{Optimizers}
    object class.
    For example, the \xmlAttr{type} attribute might be \xmlString{SPSA}, etc.
    See Section \ref{sec:Optimizers} for all the different types currently
    supported.
\end{itemize}
\nb For Multi-Run, either one \xmlNode{Sampler} or one \xmlNode{Optimizer} is required.
\item \xmlNode{SolutionExport}, \xmlDesc{string, optional (Sampler) or required (Optimizer) parameter} identifies
an entity to be used for exporting key information coming from the
\textbf{Sampler} or \textbf{Optimizer} object during the simulation. This node is \textbf{Required}
when an  \textbf{Optimizer} is used.
This XML node accepts the following attributes:
\begin{itemize}
  \item \xmlAttr{class}, \xmlDesc{required string attribute}, main object class
    type.
    This string corresponds to the tag of the main object's type used in the
    input.
    For this role, only \xmlString{DataObjects} can be used.
  \item \xmlAttr{type}, \xmlDesc{required string attribute}, the actual entity
    type.
    This attribute needs to specify the object type within the \texttt{DataObjects}
    object class.
    For example, the \xmlAttr{type} attribute might be \xmlString{PointSet},
    \xmlString{HistorySet}, etc. \\
    \nb Whether or not it is possible to export the \textbf{Sampler} solution
    depends on the \xmlAttr{type}.
    Currently, only the Samplers in the \xmlString{Adaptive} category and all Optimizers will
    export their solution into a \xmlNode{SolutionExport} entity. For Samplers, the \xmlNode{Outputs} node
    in the \texttt{DataObjects} needs to contain the goal \xmlNode{Function} name.
    For example, if  \xmlNode{Sampler} is of type \xmlString{Adaptive}, the
    \xmlNode{SolutionExport} needs to be of type \xmlString{PointSet} and
    it will contain the coordinates, in the input space, that belong to the
    ``Limit Surface''. For Optimizers, the
    \xmlNode{SolutionExport} needs to be of type \xmlString{HistorySet} and it will contains all the optimization trajectories, each as a history, that record how the variables are updated along each optimization trajectory.
\end{itemize}
\item \xmlNode{Output}, \xmlDesc{string, required parameter} identifies an
entity to be used as output for this step.
This XML node recognizes the following attributes:
\begin{itemize}
  \item \xmlAttr{class}, \xmlDesc{required string attribute}, main object class
    type.
    This string corresponds to the tag of the main object's type used in the
    input.
    For this role, only \xmlString{DataObjects}, \xmlString{Databases}, and
    \xmlString{OutStreams} may be used.
  \item \xmlAttr{type}, \xmlDesc{required string attribute}, the actual entity
    type.
    This attribute specifies the object type within the main object class.
    For example, if the \xmlAttr{class} attribute is \xmlString{DataObjects}, the
    \xmlAttr{type} attribute might be \xmlString{PointSet}.
\end{itemize}
\nb The number of \xmlNode{Output} nodes is unlimited.
\end{itemize}

Example:
\begin{lstlisting}[style=XML,morekeywords={pauseAtEnd,sleepTime,class}]
<Steps>
  ...
  <MultiRun name='StepName1' pauseAtEnd='False' sleepTime='0.01'>
    <Input class='Files' type=''>anInputFile.i</Input>
    <Input class='Files' type=''>aFile</Input>
    <Sampler class='Samplers' type = 'Grid'>aGridName</Sampler>
    <Model class='Models' type='Code'>aCode</Model>
    <Output class='Databases' type='HDF5'>aDatabase</Output>
    <Output class='DataObjects' type='History'>aData</Output>
  </MultiRun >
  <MultiRun name='StepName2' pauseAtEnd='True' sleepTime='0.02'>
    <Input   class='Files' type=''>anInputFile.i</Input>
    <Input   class='Files' type=''>aFile</Input>
    <Sampler class='Samplers' type='Adaptive'>anAS</Sampler>
    <Model   class='Models' type='Code'>aCode</Model>
    <Output  class='Databases' type='HDF5'>aDatabase</Output>
    <Output  class='DataObjects' type='History'>aData</Output>
    <SolutionExport class='DataObjects' type='PointSet'>
      aTPS
    </SolutionExport>
  </MultiRun>
  ...
</Steps>
\end{lstlisting}

%%%%%%%%%%%%%%%%%%%%
%%%%%     IOStep     %%%%%
%%%%%%%%%%%%%%%%%%%%
\subsection{IOStep}
\label{subsec:stepIOStep}
As the name suggests, the \textbf{IOStep} is the step where the user can perform
input/output operations among the different I/O entities available in RAVEN.
%
This step type is used to:
\begin{itemize}
 \item construct/update a \textit{Database} from a \textit{DataObjects} object, and
   vice versa;
 \item construct/update a \textit{DataObject} from a
   \textit{CSV} file contained in a directory;
 \item construct/update a \textit{Database} or a \textit{DataObjects} object from
   \textit{CSV} files contained in a directory;
 \item stream the content of a \textit{Database} or a \textit{DataObjects} out through
   an \textbf{OutStream} object (see section \ref{sec:outstream});
 \item store/retrieve a \textit{ROM} or  \textit{ExternalModel} to/from an external \textit{File} using Pickle module
 of Python. This function can be used to create and store ExternalModel or mathematical model (ROM) of fast solution
 trained to predict a response of interest of a physical system. These models can be
recovered in other simulations or used to evaluate the response of a physical system
 in a Python program by the implementing of the Pickle module.
\end{itemize}

%
The specifications of this type of step must be defined within an
\xmlNode{IOStep} XML block.
%
This XML node can accept the following attributes:
\vspace{-5mm}
\begin{itemize}
\itemsep0em
\item \xmlAttr{name}, \xmlDesc{required string attribute}, user-defined name of
  this Step.
  \nb As for the other objects, this is the name that can be used to refer to
    this specific entity in the \xmlNode{RunInfo} block, under the
    \xmlNode{Sequence} node.
\item \xmlAttr{pauseAtEnd}, \xmlDesc{optional boolean/string attribute (case insensitive)}, if True
  (True values = True, yes, y, t), the code will pause at the end of
  the step, waiting for a user signal to continue. This is used in case one or
  more of the \textbf{Outputs} are of type \textbf{OutStreams}.
  For example, it can be used when an \textbf{OutStreams} of type
  \textbf{Plot} is output to the screen. Thus, allowing the user to interact
  with the \textbf{Plot} (e.g. rotate the figure, change the scale, etc.).
\default{False}.
\item \xmlAttr{fromDirectory}, \xmlDesc{optional string attribute}, The directory
  where the input files can be found when loading data from a file or series of
  files directly into a DataObject.
\end{itemize}
\vspace{-5mm}
In the \xmlNode{IOStep} input block, the user specifies the objects that need to
be used for the different allowable roles.
This step accepts the following roles:
\begin{itemize}
\item \xmlNode{Input}, \xmlDesc{string, required parameter}, names an entity
  that is going to be used as a source (input) from which the information needs
  to be extracted.
  This XML node recognizes the following attributes:
\begin{itemize}
  \item \xmlAttr{class}, \xmlDesc{required string attribute}, main object class
    type.
    This string corresponds to the tag of the main object's type used in the
    input.
    As already mentioned, the allowable main classes are \xmlString{DataObjects},
    \xmlString{Databases}, \xmlString{Models} and \xmlString{Files}.
  \item \xmlAttr{type}, \xmlDesc{required string attribute}, the actual entity
    type.
    This attribute needs to specify the object type within the main object
    class.
    For example, if the  \xmlAttr{class} attribute is \xmlString{DataObjects}, the
    \xmlAttr{type} attribute might be \xmlString{PointSet}. If the  \xmlAttr{class} attribute is \xmlString{Models}, the
    \xmlAttr{type} attribute must be \xmlString{ROM} or \xmlString{ExternalModel} and if the  \xmlAttr{class} attribute is \xmlString{Files}, the
    \xmlAttr{type} attribute must be \xmlString{ }
\end{itemize}
\item \xmlNode{Output}, \xmlDesc{string, required parameter} names an entity to
  be used as the target (output) where the information extracted in the input
  will be stored.
  This XML node needs to contain the following attributes:
\begin{itemize}
  \item \xmlAttr{class}, \xmlDesc{required string attribute}, main object class
    type.
    This string corresponds to the tag of the main object's type used in the
    input.
    The allowable main classes are \xmlString{DataObjects}, \xmlString{Databases}, \xmlString{OutStreams}, \xmlString{Models} and \xmlString{Files}.
  \item \xmlAttr{type}, \xmlDesc{required string attribute}, the actual entity
    type.
    This attribute specifies the object type within the main object class.
    For example, if the \xmlAttr{class} attribute is
    \xmlString{OutStreams}, the \xmlAttr{type} attribute might be
    \xmlString{Plot}.
\end{itemize}
\end{itemize}
This step acts as a ``transfer network'' among the different RAVEN storing
(or streaming) objects.
%
The number of \xmlNode{Input} and \xmlNode{Output} nodes is unlimited, but
should match.
%
This step assumes a 1-to-1 mapping (e.g. first \xmlNode{Input} is going to be
used for the first \xmlNode{Output}, etc.).
\\
\nb This 1-to-1 mapping is not present when \xmlNode{Output} nodes are of
\xmlAttr{class}\\\xmlString{OutStreams}, since  \textbf{OutStreams}
objects are already linked to a Data object in the relative RAVEN input block.
%
In this case, the user needs to provide all of the
``DataObjects'' objects linked to the OutStreams objects (see the example
below) in the \xmlNode{Input} nodes.
\begin{lstlisting}[style=XML,morekeywords={class}]
<Steps>
  ...
  <IOStep name='OutStreamStep'>
    <Input class='DataObjects' type='HistorySet'>aHistorySet</Input>
    <Input class='DataObjects' type='PointSet'>aTPS</Input>
    <Output class='OutStreams' type='Plot'>plot_hist
    </Output>
    <Output class='OutStreams' type='Print'>print_hist
    </Output>
    <Output class='OutStreams' type='Print'>print_tps
    </Output>
    <Output class='OutStreams' type='Print'>print_tp
    </Output>
  </IOStep>
  ...
  <IOStep name='PushDataObjectsIntoDatabase'>
    <Input  class='DataObjects'     type='HistorySet'>aHistorySet</Input>
    <Input  class='DataObjects'     type='PointSet'>aTPS</Input>
    <Output class='Databases' type='NetCDF'>aDatabase</Output>
    <Output class='Databases' type='HDF5'>aDatabase</Output>
  </IOStep>
  ...
</Steps>
\end{lstlisting}
%

A summary of the objects that can go from/to other objects is shown in Table \ref{tab:IOSTEP}:
\begin{table}[h!]
  \centering
  \begin{tabular}{l|l|l}
    \xmlNode{Input} & \xmlNode{Output} & Resulting behavior  \\ \hline
    DataObject & Database   & Store to On-Disk Database      \\
               & OutStream  & Print or Plot Data             \\ \hline
    Database   & DataObject & Load from On-Disk Database     \\ \hline
    File       & DataObject & Load from On-Disk CSV          \\
                & ExternalModel        & Load On-Disk Serialized ExternalModel    \\
               & ROM        & Load On-Disk Serialized ROM    \\ \hline
    ROM        & DataObject & Print ROM Metadata to CSV, XML \\
               & File       & Serialize ROM to Disk   \\ \hline
    ExternalModel     & File       & Serialize ExternalModel to Disk \\
  \end{tabular}
  \caption{Object options for \xmlNode{IOStep} operations}
  \label{tab:IOSTEP}
\end{table}

As already mentioned, the \xmlNode{IOStep} can be used to export (serialize) a ROM or ExternalModel
in a binary file. To use the exported ROM or ExternalModel in an external Python (or
Python-compatible) code, the RAVEN framework must be present in end-user machine.
The main reason for this is that the \textit{Pickle} module uses the class definitions to template
the reconstruction of the serialized object in memory.
\\ In order to faciliate the usage of the serialized ROM or ExternalModel in an external Python code, the RAVEN
team provided a utility class contained in :
\begin{lstlisting}[language=bash]
 ./raven/scripts/externalROMloader.py
\end{lstlisting}
An example of how to use this utility class to load and use a serialized ROM (already trained) or ExternalModel is reported below:
%
Example Python Function:
\begin{lstlisting}[language=python]
from externalROMloader import ravenROMexternal
import numpy as np
rom = ravenROMexternal("path_to_pickled_rom/ROM.pk",
                                     "path_to_RAVEN_framework")
request = {"x1":np.atleast_1d(Value1),"x2":np.atleast_1d(Value2)}
eval  = rom.evaluate(request)
print str(eval)
\end{lstlisting}

The module above can also be used to evaluate a ROM or ExternalModel from input file:
\begin{lstlisting}[language=bash]
  python ./raven/scripts/externalROMloader.py input_file.xml
\end{lstlisting}
The input file has the following format:
\begin{lstlisting}[style=XML,morekeywords={class}]
<?xml version="1.0" ?>
<external_rom>
  <RAVENdir>path_to_RAVEN_framework</RAVENdir>
  <ROMfile>ath_to_pickled_rom/ROM.pk</ROMfile>
  <evaluate>
    <x1>0. 1. 0.5</x1>
    <x2>0. 0.4 2.1</x2>
  </evaluate>
  <inspect>true</inspect>
  <outputFile>output_file_name</outputFile>
</external_rom>
\end{lstlisting}
The output of the above command would look like as follows:
\begin{lstlisting}[style=XML,morekeywords={class}]
<?xml version="1.0" ?>
<UROM>
  <settings>
    <Target>ans</Target>
    <name>UROM</name>
    <IndexSet>TensorProduct</IndexSet>
    <Features>[u'x1' u'x2']</Features>
    <PolynomialOrder>2</PolynomialOrder>
  </settings>
  <evaluations>
    <evaluation realization="1">
      <x2>0.0</x2>
      <x1>0.0</x1>
      <ans>-3.1696867353e-14</ans>
    </evaluation>
    <evaluation realization="2">
      <x2>0.4</x2>
      <x1>1.0</x1>
      <ans>1.4</ans>
    </evaluation>
    <evaluation realization="3">
      <x2>2.1</x2>
      <x1>0.5</x1>
      <ans>2.6</ans>
    </evaluation>
  </evaluations>
</UROM>
\end{lstlisting}





%%%%%%%%%%%%%%%%%%%%
%%%%%   ROM    %%%%%
%%%%%%%%%%%%%%%%%%%%
\subsection{RomTrainer}
\label{subsec:stepRomTrainer}
The \textbf{RomTrainer} step type performs the training of a Reduced Order
Model (aka Surrogate Mode).
%
The specifications of this step must be defined within a \xmlNode{RomTrainer}
block.
%
This XML node accepts the attributes:
\vspace{-5mm}
\begin{itemize}
\itemsep0em
\item \xmlAttr{name}, \xmlDesc{required string attribute}, user-defined name of
this step.
\nb As for the other objects, this is the name that can be used to refer to this
specific entity in the \xmlNode{RunInfo} block under \xmlNode{Sequence}.
\end{itemize}
\vspace{-5mm}
In the \xmlNode{RomTrainer} input block, the user will specify the objects
needed for the different allowable roles.
%
This step accepts the following roles:
\begin{itemize}
\item \xmlNode{Input}, \xmlDesc{string, required parameter} names an entity to
be used as a source (input) from which the information needs to be extracted.
This XML node accepts the following attributes:
  \begin{itemize}
  \item \xmlAttr{class}, \xmlDesc{required string attribute}, main object class
    type.
    This string corresponds to the tag of the main object's type used in the
    input.
    The only allowable main class is \xmlString{DataObjects}.
  \item \xmlAttr{type}, \xmlDesc{required string attribute}, the actual entity
    type.
    This attribute specifies the object type within the main object class.
    For example, the \xmlAttr{type} attribute might be \xmlString{PointSet}.
    \nb Depending on which type of \xmlString{DataObjects} is used, the ROM
    will be a Static or Dynamic (i.e. time-dependent) model. This implies that
    both \xmlString{PointSet} and \xmlString{HistorySet} are allowed (but not
    \xmlString{DataSet} yet).
  \end{itemize}
\item \xmlNode{Output}, \xmlDesc{string, required parameter}, names a ROM entity
  that is going to be trained.
  This XML node recognizes the following attributes:
  \begin{itemize}
  \item \xmlAttr{class}, \xmlDesc{required string attribute}, main object class
    type.
    This string corresponds to the tag of the main objects type used in the
    input.
    The only allowable main class is \xmlString{Models}.
  \item \xmlAttr{type}, \xmlDesc{required string attribute}, the actual entity
    type.
    This attribute needs to specify the object type within the main object
    class.
    The only type accepted here is, currently, \xmlString{ROM}.
  \end{itemize}
\end{itemize}

Example:
\begin{lstlisting}[style=XML,morekeywords={class}]
<Steps>
  ...
  <RomTrainer name='aStepNameStaticROM'>
    <Input  class='DataObjects'  type='PointSet'>aPS</Input>
    <Output class='Models'         type='ROM'     >aROM</Output>
  </RomTrainer>
    <RomTrainer name='aStepNameTimeDependentROM'>
    <Input    class='DataObjects'  type='HistorySet'>aHS</Input>
    <Output class='Models'           type='ROM'       >aTimeDepROM</Output>
  </RomTrainer>
  ...
</Steps>
\end{lstlisting}
%%%%%%%%%%%%%%%%%%%%
 %%%%% PostProcess %%%%%
%%%%%%%%%%%%%%%%%%%%
\subsection{PostProcess}
\label{subsec:stepPostProcess}
The \textbf{PostProcess} step is used to post-process data or manipulate RAVEN
entities.
%
It is aimed at performing a single action that is employed by a
\textbf{Model} of type \textbf{PostProcessor}.
%

The specifications of this type of step is defined within a
\xmlNode{PostProcess} XML block.
%
This XML node specifies the following attributes:
\vspace{-5mm}
\begin{itemize}
\itemsep0em
\item \xmlAttr{name}, \xmlDesc{required string attribute}, user-defined name of
  this Step.
  \nb As for the other objects, this is the name that is used to refer to
  this specific entity in the \xmlNode{RunInfo} block under the
  \xmlNode{Sequence} node.
\item \xmlAttr{pauseAtEnd}, \xmlDesc{optional boolean/string attribute (case insensitive)}, if True
  (True values = True, yes, y, t), the code will pause at the end of
  the step, waiting for a user signal to continue. This is used in case one or
  more of the \textbf{Outputs} are of type \textbf{OutStreams}.
  For example, it can be used when an \textbf{OutStreams} of type
  \textbf{Plot} is output to the screen. Thus, allowing the user to interact
  with the \textbf{Plot} (e.g. rotate the figure, change the scale, etc.).
  \default{False}.
\end{itemize}
\vspace{-5mm}
In the \xmlNode{PostProcess} input block, the user needs to specify the objects
needed for the different allowable roles.
%
This step accepts the following roles:
\begin{itemize}
\item \xmlNode{Input}, \xmlDesc{string, required parameter}, names an entity to
  be used as input for the model specified in this step.
  This XML node accepts the following attributes:
\begin{itemize}
  \item \xmlAttr{class}, \xmlDesc{required string attribute}, main object class
    type.
    This string corresponds to the tag of the main object's type used in the
    input.
    For example, \xmlString{Files}, \xmlString{DataObjects}, \xmlString{Databases},
    etc.
  \item \xmlAttr{type}, \xmlDesc{required string attribute}, the actual entity
    type.
    This attribute specifies the object type within the main object class.
    For example, if the  \xmlAttr{class} attribute is \xmlString{DataObjects},
    the \xmlAttr{type} attribute might be \xmlString{PointSet}.
    \nb The class \xmlString{Files} has no type (i.e.
    \xmlAttr{type}\textbf{\texttt{=''}}).
\end{itemize}
\nb The \xmlAttr{class} and, consequently,  the \xmlAttr{type} usable for this
role depends on the particular type of \textbf{PostProcessor} being used.
In addition, the user can specify as many \xmlNode{Input} nodes as needed by the
model.
\item \xmlNode{Model}, \xmlDesc{string, required parameter}, names an entity to
be used as a model for this step.
This XML node recognizes the following attributes:
\begin{itemize}
  \item \xmlAttr{class}, \xmlDesc{required string attribute}, main object class
    type.
    This string corresponds to the tag of the main object's type used in the
    input.
    For this role, only \xmlString{Models} can be used.
  \item \xmlAttr{type}, \xmlDesc{required string attribute}, the actual entity
    type.
    This attribute needs to specify the object type within the
    \xmlString{Models} object class.
    The only type accepted here is \xmlString{PostProcessor}.
\end{itemize}
\item \xmlNode{Output}, \xmlDesc{string, required/optional parameter}, names an
  entity to be used as output for the PostProcessor.
  The necessity of this XML block and the types of entities that can be used as
  output depend on the type of \textbf{PostProcessor} that has been used as a
  \textbf{Model} (see section \ref{sec:models_postProcessor}).
  This XML node specifies the following attributes:
\begin{itemize}
  \item \xmlAttr{class}, \xmlDesc{required string attribute}, main object class
    type.
    This string corresponds to the tag of the main object's type used in the
    input.
  \item \xmlAttr{type}, \xmlDesc{required string attribute}, the actual entity
    type.
    This attribute specifies the object type within the main object class.
    For example, if the  \xmlAttr{class} attribute is \xmlString{DataObjects}, the
    \xmlAttr{type} attribute might be \xmlString{PointSet}.
\end{itemize}
\nb The number of \xmlNode{Output} nodes is unlimited.
\end{itemize}

Example:
\begin{lstlisting}[style=XML,morekeywords={class}]
<Steps>
  ...
  <PostProcess name='PP1'>
      <Input  class='DataObjects'  type='PointSet' >aData</Input>
      <Model  class='Models' type='PostProcessor'>aPP</Model>
      <Output class='Files'  type=''>anOutputFile</Output>
  </PostProcess>
  ...
</Steps>
\end{lstlisting}
