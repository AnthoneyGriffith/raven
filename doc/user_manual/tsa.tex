\newcommand{\tsaList}[0]
{
In addition, any number of the following TSA algorithms may be included as subnodes:

\begin{itemize}
  %%%%% Wavelet %%%%%
\item \xmlNode{Wavelet} performs a discrete wavelet transform on time-dependent
  data. Note: This TSA module requires pywavelets to be installed within your
  python environment.
  \begin{itemize}
  \item \xmlAttr{target}, \xmlDesc{comma sperated string, require field},
    indicates which target signals should be trained as part of this entity using this
    TSA algorithm.
  \item \xmlNode{family}, \xmlDesc{string, required field}, indicates which family
    of wavelets to use.

    There are several possible families to choose from, and most families contain
    more than one variation. For more information regarding the wavelet families,
    refer to the Pywavelets documentation.

    Possible values are:
    \begin{itemize}
    \item \textbf{haar family}: haar
    \item \textbf{db family}: db1, db2, db3, db4, db5, db6, db7, db8, db9, db10, db11,
      db12, db13, db14, db15, db16, db17, db18, db19, db20, db21, db22, db23,
      db24, db25, db26, db27, db28, db29, db30, db31, db32, db33, db34, db35,
      db36, db37, db38
    \item \textbf{sym family}: sym2, sym3, sym4, sym5, sym6, sym7, sym8, sym9, sym10,
      sym11, sym12, sym13, sym14, sym15, sym16, sym17, sym18, sym19, sym20
    \item \textbf{coif family}: coif1, coif2, coif3, coif4, coif5, coif6, coif7, coif8,
      coif9, coif10, coif11, coif12, coif13, coif14, coif15, coif16, coif17
    \item \textbf{bior family}: bior1.1, bior1.3, bior1.5, bior2.2, bior2.4, bior2.6,
      bior2.8, bior3.1, bior3.3, bior3.5, bior3.7, bior3.9, bior4.4, bior5.5,
      bior6.8
    \item \textbf{rbio family}: rbio1.1, rbio1.3, rbio1.5, rbio2.2, rbio2.4, rbio2.6,
      rbio2.8, rbio3.1, rbio3.3, rbio3.5, rbio3.7, rbio3.9, rbio4.4, rbio5.5,
      rbio6.8
    \item \textbf{dmey family}: dmey
    \item \textbf{gaus family}: gaus1, gaus2, gaus3, gaus4, gaus5, gaus6, gaus7, gaus8
    \item \textbf{mexh family}: mexh
    \item \textbf{morl family}: morl
    \item \textbf{cgau family}: cgau1, cgau2, cgau3, cgau4, cgau5, cgau6, cgau7, cgau8
    \item \textbf{shan family}: shan
    \item \textbf{fbsp family}: fbsp
    \item \textbf{cmor family}: cmor
    \end{itemize}
  \end{itemize}
  %%%%% PolynomialRegression %%%%%
\item \xmlNode{PolynomialRegression} fits time-series data using a polynomial function of degree one or greater.

  \xmlNode{PolynomialRegression} has the following attributes:
  \begin{itemize}
    \item \xmlAttr{target}, \xmlDesc{comma seperated string, required field}, indicates which
        target signals should be trained as part of this entity using this TSA algorithm.
  \end{itemize}

  \xmlNode{PolynomialRegression} has the following subnodes:
  \begin{itemize}
  \item \xmlNode{degree}, \xmlDesc{integer, required field}, indicates which
    degree of polynomial to fit to the presented data.
  \end{itemize}

  %%%%% FOURIER %%%%%
  \item \xmlNode{Fourier} uses regression to fit requested Fourier bases by their amplitudes to
    deterministically match the training signal. Fourier signals are defined with the following form:
    \begin{equation*}
      F_m(t) = C_m\sin\left( \frac{2\pi}{k_m}t + \phi_m\right)
    \end{equation*}
    where $m$ indexes a particular base period $k_m$, $C_m$ is the amplitude of this Fourier base in the
    training signal, and $\phi_m$ is the phase shift of this Fourier base in the training signal.

    \xmlNode{Fourier} has the following parameters:
    \begin{itemize}
      \item \xmlAttr{target}, \xmlDesc{comma seperated string, required field}, indicates which
        target signals should be trained as part of this entity using this TSA algorithm.
      \item \xmlAttr{seed}, \xmlDesc{integer, optional}, provides a static seed to be used for
        random number generation in this algorithm. Unused for the Fourier TSA algorithm.
    \end{itemize}
    \xmlNode{Fourier} further has the following subnodes:
    \begin{itemize}
      \item \xmlNode{periods}, \xmlDesc{comma seperated floats, required field}, indicates which base
        periods whose Fourier reperesentations should be fit to the training signal. For example, in
        a signal with hourly measurements, selecting the period \xmlString{12, 24} would fit the daily
        (24-hour) and half-daily (12-hour) periodic trends.
    \end{itemize}

  %%%%% ARMA %%%%%
  \item \xmlNode{ARMA} characterizes the signal using Auto-Regressive and Moving Average
    coefficients to stochastically fit the training signal.
    The ARMA representation has the following form:
    \begin{equation*}
      A_t = \sum_{i=1}^P \phi_i A_{t-i} + \epsilon_t + \sum_{j=1}^Q \theta_j \epsilon_{t-j},
    \end{equation*}
    where $t$ indicates a discrete time step, $\phi$ are the signal lag (or auto-regressive)
    coefficients, $P$ is the number of signal lag terms to consider, $\epsilon$ is a random noise
    term, $\theta$ are the noise lag (or moving average) coefficients, and $Q$ is the number of
    noise lag terms to consider. The ARMA algorithms are developed in RAVEN using the
    \texttt{statsmodels} Python library.

    \xmlNode{ARMA} has the following parameters:
    \begin{itemize}
      \item \xmlAttr{target}, \xmlDesc{comma seperated string, required field}, indicates which
        target signals should be trained as part of this entity using this TSA algorithm.
      \item \xmlAttr{seed}, \xmlDesc{integer, optional}, provides a static seed to be used for
        random number generation in this algorithm. This applies both to the training and sampling of this
        algorithm.
      \item \xmlAttr{reduce\_memory}, \xmlDesc{boolean, optional field}, activates a lower memory
        usage ARMA training. This does tend to result
        in a slightly slower training time, at the benefit of lower memory usage. For
        example, in one 1000-length history test, low memory reduced memory usage by 2.3
        MiB (80\%), but increased training time by 0.4 seconds (20\%). No change in results has been
        observed switching between modes. Note that the ARMA must be
        retrained to change this property; it cannot be applied to serialized ARMAs.
        \default{False}
    \end{itemize}
    \xmlNode{ARMA} further has the following subnodes:
    \begin{itemize}
      \item \xmlNode{SignalLag}, \xmlDesc{integer, required field}, number of signal lag terms to
        include in the autoregression term.
      \item \xmlNode{NoiseLag}, \xmlDesc{integer, required field}, number of noise lag terms to
        include in the moving average term.
    \end{itemize}
\end{itemize}
}
