%%%%%%%%%%%%%%%%%%%%%%%%%%%%%%%%%%%%%%%%%%%%%%%%%%%%%%
%%%%%%%%%%%%%%%%% Neutrino Interface %%%%%%%%%%%%%%%%%%
%%%%%%%%%%%%%%%%%%%%%%%%%%%%%%%%%%%%%%%%%%%%%%%%%%%%%%
\subsection{Neutrino Interface}
\label{subsec:neutrinoInterface}
This section covers the input specification for running Neutrino through RAVEN. It is important to notice
that this explanation assumes that the reader already knows how to use Neutrino. The existing inteface can be used to modify the particle size.
However, the interface can be modified to alter other parameters by using a similar method to the
existing particle size modification included in the interface.

\subsubsection{Files}
In the \xmlNode{Files} section, as specified before, all the files needed for the code to
run should be specified. In the case of Neutrino, the file needed is the following:
\begin{itemize}
  \item Neutrino input file with file extension `.nescene';
\end{itemize}
The Neutrino input file name must be NeutrinoInput.nescene. Otherwise, the Neutrino interface must be modified.
%
Example:
\begin{lstlisting}[style=XML]
  <Files>
    <Input name="neutrinoInput" type="">NeutrinoInput.nescene</Input>
  </Files>
\end{lstlisting}

%%%%%%%%%%%%%%%%%%%%%%%%%%%%%%%%%%%%%%%%%%%%%%%%%%%%%X
\subsubsection{Models}
In the \xmlNode{Models} block, the Neutrino executable needs to be specified. The entire path to the Neutrino executable must be included.
 Here is a standard example of what
can be used:
\begin{lstlisting}[style=XML]
  <Models>
    <Code name="neutrinoCode" subType="Neutrino">
      <executable>"C:\Program Files\Neutrino_02_22_19\Neutrino.exe"</executable>
    </Code>
  </Models>
\end{lstlisting}

The \xmlNode{Code} XML node contains the information needed to execute the specific External Code. This
XML node accepts the following attributes:
\begin{itemize}
  \item \xmlAttr{name}, \xmlDesc{required string attribute}, user-defined identifier of this model.
    \nb As with other objects, this identifier can be used to reference this specific entity from other input
    blocks in the XML.
  \item \xmlAttr{subType}, \xmlDesc{required string attribute}, specifies the code that needs to be
    associated to this Model.
\end{itemize}
This model can be initialized with the following children:
\begin{itemize}
  \item \xmlNode{executable}, \xmlDesc{string, required field}, specifies the path of the executable to
    be used; \nb Either an absolute or relative path can be used.
\end{itemize}


%%%%%%%%%%%%%%%%%%%%%%%%%%%%%%%%%%%%%%%%%%%%%%%%%%%%%%
\subsubsection{Distributions}
The \xmlNode{Distributions} block defines the distributions that are going to be used for the sampling
of the variables defined in the \xmlNode{Samplers} block. For all the possible distributions and all
their possible inputs, please see the chapter about Distributions (see~\ref{sec:distributions}). Here is an example of a
Uniform distribution:
\begin{lstlisting}[style=XML]
  <Distributions>
    <Uniform name="uni">
        <lowerBound>0.1</lowerBound>
        <upperBound>0.2</upperBound>
    </Uniform>
  </Distributions>
\end{lstlisting}
%
%%%%%%%%%%%%%%%%%%%%%%%%%%%%%%%%%%%%%%%%%%%%%%%%%%%%%%
\subsubsection{Samplers}
The \xmlNode{Samplers} block defines the variables to be sampled. After defining a sampling scheme, the variables to be sampled and
their distributions are identified in the \xmlNode{variable} blocks.
The \xmlAttr{name} must be formatted according to the Neutrino parameter name, which for the particle size is
'ParticleSize'.
An example of a \xmlNode{Samplers} block is given below:
\begin{lstlisting}[style=XML]
 <Samplers>
    <MonteCarlo name="myMC">
      <samplerInit>
        <limit>5</limit>
      </samplerInit>
      <variable name='ParticleSize'>
        <distribution>uni</distribution>
      </variable>
    </MonteCarlo>
  </Samplers>
\end{lstlisting}

%%%%%%%%%%%%%%%%%%%%%%%%%%%%%%%%%%%%%%%%%%%%%%%%%%%%%%
\subsubsection{Steps}
In this section, the \xmlNode{MultiRun} will be used. As shown in the following, a Neutrino
input file is listed in \xmlNode{Files} and is linked here using \xmlNode{Input}, the \xmlNode{Model}
and \xmlNode{Sampler} defined in previous sections will be used in this \xmlNode{MultiRun}. The
outputs will be saved in the \textbf{DataObject} 'resultPointSet'.

\begin{lstlisting}[style=XML]
  <Steps>
    <MultiRun name="run">
      <Input class="Files" type="">neutrinoInput</Input>
      <Model class="Models" type="Code">neutrinoCode</Model>
      <Sampler class="Samplers" type="MonteCarlo">myMC</Sampler>
      <Output class="DataObjects" type="PointSet">resultPointSet</Output>
    </MultiRun>
  </Steps>
\end{lstlisting}

%%%%%%%%%%%%%%%%%%%%%%%%%%%%%%%%%%%%%%%%%%%%%%%%%%%%%%
\subsubsection{Output File Conversion}
The Neutrino measurement field output is a CSV output. However, labels must be added to the Neutrino output
and it must be moved for RAVEN. These are both done in the Neutrino interface. The labels that are added to the
output file are 'time' and 'result'. These labels would be used in the \xmlNode{DataObjects} specification.
If different labels are wanted, they would need to be changed directly in the Neutrino interface.

%%%%%%%%%%%%%%%%%%%%%%%%%%%%%%%%%%%%%%%%%%%%%%%%%%%%%%
\subsubsection{Additional Information}
The Neutrino interface is used to alter the particle size by modifying the Neutrino input file. The Neutrino interface
searches for the default SPH solver parameter name: `\texttt{NIISphSolver\_1}'. If the SPH solver name is changed in the
Neutrino input file, the Neutrino interface must also be changed. Similarly, the Neutrino interface searches for the
output in the default Measurement field name: `\texttt{MeasurementField\_1}'. Again, this would need to modified in the interface
if the measurement field name was changed.

