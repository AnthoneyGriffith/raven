%%%%%%%%%%%%%%%%%%%%%%%
%%% SCALE INTERFACE %%%
%%%%%%%%%%%%%%%%%%%%%%%
\subsection{SCALE Interface}
This section presents the main aspects of the interface between RAVEN and SCALE system,
the consequent RAVEN input adjustments and the modifications of the SCALE
files required to run the two coupled codes.
\\ \textcolor{red}{
\textbf{\textit{\nb Considering the large amount of SCALE sequences, this interface is
currently limited in driving the following SCALE calculation codes:}}
\begin{itemize}
  \item \textbf{\textit{ORIGEN}}
  \item \textbf{\textit{TRITON (using NEWT as transport solver)}}
\end{itemize}
}

In the following sections a short explanation on how to use RAVEN coupled with SCALE is reported.

%%%%%%%%%%%%%%%%%%%%%%%%%%%%%%%%%%%%%%%%%%%%%%%%%%%%%%%%%

\subsubsection{Models}
As for any other Code, in order to activate the SCALE interface, a  \xmlNode{Code} XML node needs to be inputted, within the
main XML node \xmlNode{Models}.
\\The  \xmlNode{Code} XML node contains the
information needed to execute the specific External Code.

\attrsIntro
%
\vspace{-5mm}
\begin{itemize}
  \itemsep0em
  \item \nameDescription
  \item \xmlAttr{subType}, \xmlDesc{required string attribute}, specifies the
  code that needs to be associated to this Model.
  %
  \nb See Section~\ref{sec:existingInterface} for a list of currently supported
  codes.
  %
\end{itemize}
\vspace{-5mm}

\subnodesIntro
%
\begin{itemize}
  \item \xmlNode{executable} \xmlDesc{string, required field} specifies the path
  of the executable to be used.
  %
  \nb Either an absolute or relative path can be used.
  \item \aliasSystemDescription{Code}
  %
\end{itemize}

In addition (and specifc for the SCALE interface), the  \xmlNode{Code} can contain the following optional nodes:

\begin{itemize}
  \item \xmlNode{sequence}, optional, comma separated list. In this node the user can specify a list of sequences that need to be
  executed in sequence. For example, if a TRITON calculation needs to be followed by an ORIGEN decay heat calculation the user
  would input here the sequence ``\textit{triton,origen}''. \default{triton}.
  \\\nb Currently only the following entries are supported:
    \begin{itemize}
     \item  ``\textit{triton}''
     \item  ``\textit{origen}''
     \item  ``\textit{triton,origen}''
    \end{itemize}
  \item \xmlNode{timeUOM}, optional, string. In this node the user can specify  the \textit{units} for the independent variable ``time''.
   If the outputs are exported by SCALE in a different unit (e.g days, years, etc.), the SCALE interface will convert all the different
   time scales into the unit here specified (in order to have a consistent  (and unique) time scale). Available are:
    \begin{itemize}
     \item ``\textit{s}'', seconds
     \item ``\textit{m}'', minutes
     \item ``\textit{h}'', hours
     \item ``\textit{d}'', days
     \item ``\textit{y}'', years
    \end{itemize}
    \default{s}
\end{itemize}

An example  is shown  below:
\begin{lstlisting}[style=XML]
<Models>
    <Code name="MyScale" subType="Scale">
      <executable>path/to/scalerte</executable>
      <sequence>triton,origen</sequence>
      <timeUOM>d</timeUOM>
    </Code>
</Models>
\end{lstlisting}

%%%%%%%%%%%%%%%%%%%%%%%%%%%%%%%%%%%%%%%%%%%%%%%%%%%%%%%%%%%%%%%%%%%%%%%%%%%%%%%%%%
\subsubsection{Files}
%%%%%%%%%%%%%%%%%%%%%%%%%%%%%%%%%%%%%%%%%%%%%%%%%%%%%%%%%%%%%%%%%%%%%%%%%%%%%%%%%%
The \xmlNode{Files} XML node has to contain all the files required by the particular
sequence (s) of the external code  (SCALE) to be run.
This involves not only the input file(s) (.inp) but also the auxiliary files that might be needed (e.g. binary initial compositions, etc.).
As mentioned, the current SCALE interface only supports TRITON and ORIGEN sequences. For this reason, depending on the
type of sequence (see previous section) to be run, the relative input files need to be marked with the sequence they are associated
with. This means that the type of the input file must be either ``triton'' or ``origen''. The auxiliary files that might be needed by
a particular sequence (e.g. binary initial compositions, etc.) should not be marked with any specific type (i.e. \textit{type=``''}).
Example:
\begin{lstlisting}[style=XML]
<Files>
  <Input name="triton_input" type="triton">pwr_depletion.inp</Input>
  <Input name="origen_input" type="origen">decay_calc.inp</Input>
  <Input name="binary_comp" type="">pwr_depletion.f71</Input>
</Files>
\end{lstlisting}
The files mentioned in this section
 need, then, to be placed into the working directory specified
by the \xmlNode{workingDir} node in the \xmlNode{RunInfo} XML node.

\paragraph{Output Files conversion}
Since RAVEN expects to receive a CSV file containing the outputs of the simulation, the results in the SCALE output
files need to be converted by the code interface.
\\As mentioned, the current interface \textcolor{red}{ is able to collect data from TRITON and ORIGEN sequences only}.
%% TRITON
\\The following information is collected from TRITON output:
\begin{itemize}
  \item \textit{\textbf{k-eff and k-inf time-dep information}}
  \begin{lstlisting}[basicstyle=\tiny]
  Outer   Eigenvalue Eigenvalue Max Flux   Max Flux     Max Fuel   Max Fuel     Wall   Elapsed   Iteration  CPU   Inners
 Iter. #              Delta      Delta   Location(r,g)   Delta   Location(r,g) Clock   CPU Time   CPU Time Usage Converged
 - - - - - - - - - - - - - - - - - - - - - - - - - - - - - - - - - - - - - - - - - - - - - - - - - - - - - - - - - - - -
     1    1.00000   0.000E+00 6.480E+09 (    4,252)   1.000E+00 (  614,  0) 14:16:42   89.9 s    89.9 s  92.7%    F
     2    0.35701   1.801E+00 4.149E+01 (  319,  4)   2.673E+00 ( 7035,  0) 14:18:16  182.8 s    92.9 s  98.8%    F
 k-eff =       0.94724509     Time=      0.00d Nominal conditions

   Four-Factor Estimate of k-infinity.  Fast/Thermal boundary:   0.6250 eV
      Fiss. neutrons/thermal abs. (eta):          1.279827
      Thermal utilization (f):                    0.960903
      Resonance Escape probability (p):           0.706209
      Fast-fission factor (epsilon):              1.091716
                                            --------------
      Infinite neutron multiplication             0.948143

\end{lstlisting}
   that will be converted in the following way (CSV):
   \begin{table}[h]
    \centering
    \caption{CSV transport info}
    \label{CSVkeff}
    \tabcolsep=0.11cm
    \tiny
    \begin{tabular}{|c|c|c|c|c|c|c|c|c|c|}
     time & keff       & iter\_number & keff\_delta & max\_flux\_delta & kinf     & kinf\_epsilon & kinf\_p  & kinf\_f  & kinf\_eta \\
     0.00 & 0.94724509 & 2            & 1.801E+00   & 4.149e+01        & 0.948143 & 1.091716      & 0.706209 & 0.960903 & 1.279827
    \end{tabular}
   \end{table}

  \item \textit{\textbf{material powers}}
  \begin{lstlisting}[basicstyle=\tiny]
  --- Material powers for depletion pass no.   1 (MW/MITHM) ---
       Time =     0.00 days (   0.000 y), Burnup =    0.000     GWd/MTIHM, Transport k=  0.9473

                    Total    Fractional  Mixture     Mixture       Mixture
         Mixture    Power      Power      Power    Thermal Flux  Total Flux
          Number (MW/MTIHM)    (---)   (MW/MTIHM)  n/(cm^2*sec)  n/(cm^2*sec)
            13      32.985    0.99054     32.985    5.3666e+13    1.2574e+14
             6       0.252    0.00757     N/A       2.7587e+13    9.1781e+13
         Total      33.300    1.00000
\end{lstlisting}
   that will be converted in the following way (CSV):
   \begin{table}[h]
     \centering
     \caption{CSV material powers}
     \label{CSVmatPowers}
     \tabcolsep=0.11cm
     \tiny
     \begin{tabular}{|c|c|c|c|c|c|c|c|c|c|l}
     \cline{1-10}
     time    & bu  & tot\_power\_mix\_13 & fract\_power\_mix\_13 & th\_flux\_mix\_13 & tot\_flux\_mix\_13 & tot\_power\_mix\_6 & fract\_power\_mix\_6 & th\_flux\_mix\_6 & tot\_flux\_mix\_6 &  \\ \cline{1-10}
     1.0E-06 & 0.0 & 32.985              & 0.99054               & 5.3666e+13        & 1.2574e+14         & 0.252              & 0.00757              & 2.7587e+13       & 9.1781e+13        &  \\ \cline{1-10}
     \end{tabular}
   \end{table}


 \item \textit{\textbf{nuclide/element tables}}
  \begin{lstlisting}[basicstyle=\tiny]
            | nuclide concentrations
            | time: days
      grams |    0.00e+00d
------------+--------------------
       u235 |   2.9619e+04
       u238 |   9.6993e+05
   subtotal |   1.0010e+06
      total |   1.1858e+06
\end{lstlisting}
   that will be converted in the following way (CSV):
   \begin{table}[h]
    \centering
    \caption{CSV Nuclide/element Tables}
    \label{CSVnuclideTables}
    \tabcolsep=0.11cm
    \tiny
    \begin{tabular}{|c|c|c|}
     time & u235\_conc       & u238\_conc   \\
     0.00 & 2.9619e+04  & 9.6993e+05
    \end{tabular}
   \end{table}
\end{itemize}
%% ORIGEN
The following information is collected from ORIGEN output:
\begin{itemize}
  \item \textit{\textbf{history overview}}
  \begin{lstlisting}[basicstyle=\tiny]
=========================================================================================================================
=   History overview for case 'decay' (#1/1)                                                                            =
-------------------------------------------------------------------------------------------------------------------------
   step          t0          t1          dt           t        flux     fluence       power      energy
    (-)       (sec)       (sec)         (s)         (s)   (n/cm2-s)     (n/cm2)        (MW)       (MWd)
      1  0.0000E+00  1.0000E-06  1.0000E-06  1.0000E-06  0.0000E+00  0.0000E+00  0.0000E+00  0.0000E+00
\end{lstlisting}
   that will be converted in the following way (CSV):
    \begin{table}[h]
    \centering
    \caption{CSV History Overview}
    \label{CSVhistoryOverview}
    \tabcolsep=0.11cm
    \tiny
    \begin{tabular}{|c|c|c|c|c|c|c|c|}
    \hline
     time    & t0  & t1      & dt      & flux & fluence & power & energy \\
     1.0E-06 & 0.0 & 1.0E-06 & 1.0E-06 & 0.0  & 0.0     & 0.0   & 0.0
    \end{tabular}
   \end{table}

   \item \textit{\textbf{concentration tables}}
  \begin{lstlisting}[basicstyle=\tiny]
=========================================================================================================================
=   Nuclide concentrations in watts, actinides for case 'decay' (#1/1)                                                  =
-------------------------------------------------------------------------------------------------------------------------
  (relative cutoff; integral of concentrations over time >   1.00E-04 % of integral of all concentrations over time)
.
                0.0E+00sec  1.0E-06sec
  th231       8.6167E-08  8.6167E-08
  th234       7.7763E-09  7.7763E-09
------------
  totals       4.6831E+03  4.6831E+03
=========================================================================================================================
.
.
=========================================================================================================================
=   Nuclide concentrations in watts, fission products for case 'decay' (#1/1)                                           =
-------------------------------------------------------------------------------------------------------------------------
  (relative cutoff; integral of concentrations over time >   1.00E-04 % of integral of all concentrations over time)
.
                0.0E+00sec  1.0E-06sec
  ga74        2.4264E-01  2.4264E-01
  ga75        1.8106E+00  1.8106E+00
------------
  totals       1.2266E+06  1.2266E+06
  \end{lstlisting}
  that will be converted in the following way (CSV):
   \begin{table}[h]
    \centering
    \caption{CSV Concentration Tables}
    \label{CSVconcentrationTables}
    \tabcolsep=0.11cm
    \tiny
    \begin{tabular}{|c|c|c|c|c|c|c|c|}
    \hline
     time    & ga74\_watts  & ga75\_watts      & subtotals\_fission\_products      & th231\_watts & th234\_watts & subtotals\_actinides & totals\_watts \\ \hline
     0.0E+00 & 2.4264E-01 & 1.8106E+00 & 1.2266E+06 & 8.6167E-08  & 7.7763E-09     & 4.6831E+03   & 1.2313E+06    \\
     1.0E-06 & 2.4264E-01 & 1.8106E+00 & 1.2266E+06 & 8.6167E-08  & 7.7763E-09     & 4.6831E+03  & 1.2313E+06
    \end{tabular}
   \end{table}
\end{itemize}

\textbf{Remember also that a SCALE simulation run is considered successful (i.e., the simulation did not crash) if it does not contain, in
the last 20 lines, the following message:}

\textcolor{red}{terminated due to errors}

\textbf{If the a SCALE simulation terminates with this message, the simulation is considered ``failed'', i.e., it will not be saved.}

%%%%%%%%%%%%%%%%%%%%%%%%%%%%%%%%%%%%%%%%%%%%%%%%%%%%%%%%%%%%%%%%%%%%%%%%%%%%%%%%%%
\subsubsection{Samplers or Optimizers}
In the \xmlNode{Samplers} or  \xmlNode{Optimizers} block we want to define the variables that are going
to be sampled or optimized.
%
\\The perturbation or optimization of the input of any SCALE sequence is performed using the approach detailed in the \textit{Generic Interface} section (see \ref{subsec:genericInterface}). Briefly, this approach uses
 ``wild-cards'' (placed in the original input files) for injecting the perturbed values.
 For example, if the original input file (that needs to be perturbed) is the following:
\begin{lstlisting}[language=python]
=origen
case(actual_mass){
  lib{ file="end7dec" }
  mat{ iso=[zr-95=1.0] units="moles" }
  time=[1.0] %1 day
}
end
\end{lstlisting}
and  the initial moles of ``zr-95'' need to be perturbed, a RAVEN ``wild-card'' will be defined:
\begin{lstlisting}[language=python]
=origen
case(actual_mass){
  lib{ file="end7dec" }
  mat{ iso=[zr-95=$RAVEN-zrMoles$] units="moles" }
  time=[1.0] %1 day
}
end
\end{lstlisting}

Finally, the variable \textbf{\textit{zrMoles}} needs to be specified in the specific Sampler or Optimizer that will be used:

\begin{lstlisting}[style=XML]
...
<Samplers>
  <aSampler name='aUserDefinedName' >
    <variable name='zrMoles'>
      ...
    </variable>
  </aSampler>
</Samplers>
...
<Optimizers>
  <anOptimizer name='aUserDefinedName' >
    <variable name='zrMoles'>
      ...
    </variable>
  </anOptimizer>
</Samplers>
...
\end{lstlisting}
