%%%%%%%%%%%%%%%%%%%%%%%
%%% PARCS INTERFACE %%%
%%%%%%%%%%%%%%%%%%%%%%%
\subsection{PARCS Interface}
A recent interface between RAVEN and the diffusion code PARCS was implemented.
The primary purpose of this section is to describe the interface associated
with its application using RAVEN/Generic Algorithm in solving the Core optimization problem.
A brief explanation on the components of the PARCS interface and how to use it with RAVEN
are reported in the following subsections.

%%%%%%%%%%%%%%%%%%%%%%%%%%%%%%%%%%%%%%%%%%%%%%%%%%%%%%%%%
\subsubsection{Interface components}
%%%%%%%%%%%%%%%%%%%%%%%%%%%%%%%%%%%%%%%%%%%%%%%%%%%%%%%%%%%%%%%%%%%%%%%%%%%%%%%%%%
The interface was built based on Python 3. It included three main components:
\begin{itemize}

	\item \texttt{PARCSinterface.py}: Connects and interacts with RAVEN main module
	\item \texttt{PARCSData.py}: Collects and extracts data from depletion file and pin power distribution file generated after each PARCS simulation. The list of data can be extracted are:
	\begin{itemize}
  \item Time dependent multiplication factor $k_{eff}$;
	\item Time dependent $F_Q$;
	\item Time dependent $F_\Delta H$;
	\item Time dependent critical boron concentration;
	\item Cycle length determined by the critical boron concentration being 10 ppm;
	\item Time dependent relative pin power distribution.
  \end{itemize}
  \item \texttt{SpecificParser.py}: Generates PARCS input from sample loading pattern provided by RAVEN-GA module.
\end{itemize}

%%%%%%%%%%%%%%%%%%%%%%%%%%%%%%%%%%%%%%%%%%%%%%%%%%%%%%%%%%%%%%%%%%%%%%%%%%%%%%%%%%
\subsubsection{Models}
%%%%%%%%%%%%%%%%%%%%%%%%%%%%%%%%%%%%%%%%%%%%%%%%%%%%%%%%%%%%%%%%%%%%%%%%%%%%%%%%%%
To successfully execute the code, users need to ensure that the code PARCS is included in the
\xmlNode{Models} section in a way shown below:

\begin{lstlisting}[style=XML]
  <Models>
   <Code name="MyParcs" subType="PARCS">
     <executable>...</executable>
     <sequence>parcs</sequence>
   </Code>
  </Models>
  \end{lstlisting}

%%%%%%%%%%%%%%%%%%%%%%%%%%%%%%%%%%%%%%%%%%%%%%%%%%%%%%%%%%%%%%%%%%%%%%%%%%%%%%%%%%
\subsubsection{Files}
%%%%%%%%%%%%%%%%%%%%%%%%%%%%%%%%%%%%%%%%%%%%%%%%%%%%%%%%%%%%%%%%%%%%%%%%%%%%%%%%%%
In order to execute the PARCS-RAVEN interface, user needs to provide three input files.
Although the requested input file name can be arbitrary, they need to serve similar functions as shown below:
\begin{itemize}
	\item \xmlNode{coremap.xml}: Contains initial assigned FA type for 1/8 loading pattern following an index scheme shown below since this module was specifically designed to solve the core loading pattern in 17$\times$17 PWR core.
	\item \xmlNode{Dummy.inp}: Serves as a placeholder for perturbed PARCS input deck.
	\item \xmlNode{inputgen.xml}: Contains specification on PARCS simulation and modeling for core including FA definition; cross section information.
\end{itemize}
The requested input will be identified in RAVEN input, then they will be loaded into RAVEN memory.
With the GA optimizer, a set of sampling for possible FA location mapping
will be used to update \xmlNode{coremap.xml} file. From this point, the PARCS interface will be called to
read the remaining requested inputs with the updated \xmlNode{coremap.xml} to generate input
for PARCS simulation. After the simulation is completed, the fitness function of GA
optimizer will be evaluated and used to generate the population of the next generation.
This iterative process will continue until the maximum number of generations as specified
in the RAVEN input is reached.

Furthermore, constraints for the optimization problem can be also specified by using \xmlNode{Functions}
, see Section~\ref{sec:functions} for more details

\begin{center}
  \begin{tabular}{ |p{0.5cm}|p{0.5cm}|p{0.5cm}|p{0.5cm}|p{0.5cm}|p{0.5cm}|p{0.5cm}|p{0.5cm}|p{0.5cm}| }
    \hline
    1& & & & & & & &  \\
    \hline
    2&3& & & & & & &  \\
    \hline
    4&5&6& & & & & &  \\
    \hline
    7&8&9&10& & & & &  \\
    \hline
    11&12&13&14&15& & & &  \\
    \hline
    16&17&18&19&20&21& & &  \\
    \hline
    22&23&24&25&26&27&28& & \\
    \hline
    29&30&31&32&33&34&35&36&  \\
    \hline
    37&38&39&40&41&42&43&44&45 \\
    \hline
  \end{tabular}
\end{center}

Example:
  \begin{lstlisting}[style=XML]
  <Files>
    <Input name="parcs_input" type="parcsdata">parcs-input-gen.xml</Input>
    <Input name="parcsperturb_input" type="perturb">coremap.xml</Input>
    <Input name="input" type="input">input.inp</Input>
  </Files>
  \end{lstlisting}

Example for \xmlNode{parcs-input-gen.xml}:
  \begin{lstlisting}[style=XML]
  <PARCS-input-gen>
   <THFlag> F </THFlag>
   <power> 100 </power>
   <coretype> PWR </coretype>
   <initialBoron> 1000 </initialBoron>
   <XSdir> Xsdir </XSdir>
   <Depdir> PARCSDEP </Depdir>
   <DepHistory> 1 1 1 1 18*30 </DepHistory>
   <NFA> 9 </NFA>
   <NAxial> 14 </NAxial>
   <FA_Pitch> 21.50 </FA_Pitch>
   <FA_Power> 16.3 </FA_Power>
   <Geometry> QUARTER </Geometry>
   <grid_x> 1*10.75 8*21.50 </grid_x>
   <grid_y> 1*10.75 8*21.50 </grid_y>
   <grid_z> 30.48  12*30.48  30.48 </grid_z>
   <neutmesh_x> 1*1 8*1 </neutmesh_x>
   <neutmesh_y> 1*1 8*1 </neutmesh_y>
   <BC> 0 2 0 2 2 2 </BC>
   <FA-list>
    <FA name='FA1'  FAid ='0'  type ='20' structure = '1*6 12*1 1*7 FUEL'/>
    <FA name='FA2'  FAid ='1'  type ='30' structure = '1*6 12*2 1*7 FUEL'/>
    <FA name='FA3'  FAid ='2'  type ='40' structure = '1*6 12*3 1*7 FUEL'/>
    <FA name='FA4'  FAid ='3'  type ='50' structure = '1*6 12*4 1*7 FUEL'/>
    <FA name='FA5'  FAid ='4'  type ='60' structure = '1*6 12*5 1*7 FUEL'/>
    <FA name='REF'  FAid ='5'  type ='10' structure = '1*6 12*8 1*7 REFL'/>
    <FA name='NONE' FAid ='-1' type ='00' structure = ''/>
   </FA-list>
   <XS-list>
    <XS id='1' name='xs_g200_gd_0_bp_0'/>
    <XS id='2' name='xs_g250_gd_0_bp_0'/>
    <XS id='3' name='xs_g250_gd_16_bp_0'/>
    <XS id='4' name='xs_g320_gd_0_bp_0'/>
    <XS id='5' name='xs_g320_gd_16_bp_0'/>
    <XS id='6' name='xs_gbot'/>
    <XS id='7' name='xs_gtop'/>
    <XS id='8' name='xs_grad'/>
   </XS-list>
  </PARCS-input-gen>

\end{lstlisting}

The PARCS-RAVEN interface will perturb the \xmlNode{coremap.xml} file with the defined sampled variables in RAVEN
input file, and then utilize the user provided \xmlNode{inputgen.xml} and updated \xmlNode{coremap.xml} file to generate
PARCS input file, the following is an example for the generated PARCS input file:
Example for \xmlNode{parcs-input.inp}:
\begin{lstlisting}[basicstyle=\tiny]
  !******************************************************************************
  CASEID input.inp              OECD NEA MSLB
  !******************************************************************************
  CNTL
        TH_FDBK   F
        CORE_POWER 100
        CORE_TYPE  PWR
        PPM        1000
        DEPLETION  T  1.0E-3 T
        TREE_XS    T  16 T  T  F  F  T  F  T  F  T  F  F  T  T  F
        BANK_POS   100 100 100 100 100 100  0 46.2 0
        XE_SM      1 1
        SEARCH     ppm
        XS_EXTRAP  1.0 0.3 0.8 0.2
        PIN_POWER  T
        PRINT_OPT T F T T F T F F F T  F  F  F  F  F
  PARAM
        LSOLVER  1 1 20
        NODAL_KERN     NEMMG    ! FMFD    ! FDM
        CMFD     2
        DECUSP   2
        INIT_GUESS 0
        conv_ss   1.e-6 5.e-5 1.e-3 0.001 !epseig,epsl2,epslinf,epstf
        eps_erf   0.010
        eps_anm   0.000001
        nlupd_ss  5 5 1
  GEOM
        geo_dim 9 9 14 1 1
        Rad_Conf                        !!
          60  40  60  60  60  20  20  50  10
          40  30  30  50  30  30  50  40  10
          60  30  20  50  60  50  20  10  10
          60  50  50  60  30  50  50  10  00
          60  30  60  30  50  30  10  10  00
          20  30  50  50  30  10  10  00  00
          20  50  20  50  10  10  00  00  00
          50  40  10  10  10  00  00  00  00
          10  10  10  00  00  00  00  00  00

        grid_x      1*10.75 8*21.50
        neutmesh_x  1*1 8*1
        grid_y      1*10.75 8*21.50
        neutmesh_y  1*1 8*1
        grid_z      30.48  12*30.48  30.48
        Boun_cond   0 2 0 2 2 2
        assy_type   20   1*6   12*1    1*7 FUEL
        assy_type   30   1*6   12*2    1*7 FUEL
        assy_type   40   1*6   12*3    1*7 FUEL
        assy_type   50   1*6   12*4    1*7 FUEL
        assy_type   60   1*6   12*5    1*7 FUEL
        assy_type   10   1*6   12*8    1*7 REFL


       pincal_loc

          60  40  60  60  60  20  20  50   0
          40  30  30  50  30  30  50  40   0
          60  30  20  50  60  50  20   0   0
          60  50  50  60  30  50  50   0
          60  30  60  30  50  30   0   0
          20  30  50  50  30   0   0
          20  50  20  50   0   0
          50  40   0   0   0
           0   0   0


  TH
        unif_th         0.7  600.0  300.0
  FDBK
        fa_powpit       16.3   21.50

  DEPL
        TIME_STP  1 1 1 1 18*30
        INP_HST   '../../PARCSDEP/boc_exp_fc.dep' -2 1
        PMAXS_F   1 '../../Xsdir/xs_g200_gd_0_bp_0'                 1
        PMAXS_F   2 '../../Xsdir/xs_g250_gd_0_bp_0'                 2
        PMAXS_F   3 '../../Xsdir/xs_g250_gd_16_bp_0'                 3
        PMAXS_F   4 '../../Xsdir/xs_g320_gd_0_bp_0'                 4
        PMAXS_F   5 '../../Xsdir/xs_g320_gd_16_bp_0'                 5
        PMAXS_F   6 '../../Xsdir/xs_gbot'                 6
        PMAXS_F   7 '../../Xsdir/xs_gtop'                 7
        PMAXS_F   8 '../../Xsdir/xs_grad'
\end{lstlisting}

%%%%%%%%%%%%%%%%%%%%%%%%%%%%%%%%%%%%%%%%%%%%%%%%%%%%%%%%%%%%%%%%%%%%%%%%%%%%%%%%%%
\subsubsection{Sampler/Optimizer}
%%%%%%%%%%%%%%%%%%%%%%%%%%%%%%%%%%%%%%%%%%%%%%%%%%%%%%%%%%%%%%%%%%%%%%%%%%%%%%%%%%
As mentioned in previous sections, the variables that users desire to optimize are defined in
the \xmlNode{Samplers} or  \xmlNode{Optimizers} block. In this interface, the GA optimizer is applied.
In the loading pattern optimization problem, the variables are the location of Fuel Assembly in the core layout
with the index scheme shown in this section.

For example, the original locations are specified in the \xmlNode{coremap.xml} (that needs to be perturbed).
The name of the variables in the \xmlNode{Samplers} should be the same as the one in original \xmlNode{coremap.xml}
file, namely \textbf{loc} and the associated index \textbf{index}.

Example:
\begin{lstlisting}[style=XML]
...
<Samplers>
  ...
  <variable name="loc1">
    <distribution>FA_dist</distribution>
  ...
  </variable>
</Samplers>
...
\end{lstlisting}

